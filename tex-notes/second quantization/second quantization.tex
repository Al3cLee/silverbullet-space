\documentclass{article}

% document structure
\usepackage[hidelinks]{hyperref}
\usepackage{csquotes}
\usepackage[backend=biber,style=nature]{biblatex}
\addbibresource{ref.bib}
\usepackage[margin=1in]{geometry}

% advanced lists
\usepackage{enumitem}
\newlist{steps}{enumerate}{2}
\setlist[steps]{label=\textit{Step \arabic*}., 
leftmargin=*, ref = Step \arabic*}

% math formatting
\usepackage{xcolor,dsfont,amsthm,amsfonts,mathtools,amsmath,physics}
\usepackage[mathcal]{euscript}

% fancy boxes
\usepackage{tcolorbox}
\newtcolorbox{graybox}{colback=gray!20!white, boxrule=0mm, sharp corners=all}

\theoremstyle{definition}
\newtheorem{res}{Result}[section]
\newtheorem{mot}{Motivation}[section]
\newtheorem{calc}{Calculation}[section]
\newtheorem{rem}{Remark}[section]
\newtheorem{dfn}{Definition}[section]
\newtheorem{eg}{Example}[section]
\newtheorem{recipe}{Recipe}[section]

\theoremstyle{plain}
\newtheorem{thm}{Theorem}[section]
\newtheorem{lem}{Lemma}[section]
\numberwithin{equation}{section}

\usepackage{cleveref}
\crefname{figure}{fig.}{figs.}
\crefname{equation}{eqn.}{eqns.}
\crefname{dfn}{definition}{definitions}
\crefname{mot}{motivation}{motivations}
\crefname{thm}{theorem}{theorems}
\crefname{lem}{lemma}{lemmas}
\crefname{rem}{remark}{remarks}
\crefname{res}{result}{results}
\crefname{eg}{example}{examples}
\crefname{calc}{calculation}{calculations}

\author{Wentao Li}
\begin{document}
    
\title{Second Quantization}

\maketitle

\begin{abstract}
    This note introduces key concepts in second quantization. 
    How many-body states are symmetrized and then normalized 
    to give Fock states is explained in detail. 
    The matrix elements of 1- and 2-body operators 
    under the Fock basis are calculated. 

    Prerequisites include projectors, subspaces, and 
    basic notions about quantum states and Hilbert spaces. 
    Ref.~\cite[][\S 1.2]{dupuis_field_2023} 
    is the main reference for this note. 
    Comments on other sources are given in the bibliographical notes section. 
\end{abstract}

\tableofcontents
\section{Physical subspaces and their projectors}
\begin{mot}
In 3 spatial dimensions, there are only two types of 
exchange statistics of quantum-mechanical particles. 
\footnote{
In (2+1)D there exist particles that are neither 
Fermions nor Bosons, known as ``anyons''. We shall see 
in this note that Boson/Fermion statistics is closely tied 
to permutations (which form a group). 
Similarly, anyon statistics is closely tied to braids (which also form a group).
}
These are Bosonic statistics where permutations are trivial, 
and Fermionic statistics where the wavefunction changes sign 
upon an odd permutation. 
This bears consequences for wavefunctions. 
For example, if two particles are identical, 
then the direct product state $\ket{0}\otimes \ket{1}$ becomes 
unphysical: it is neither Bosonic 
(which should be $\ket{01}+\ket{10}$) 
nor Fermionic 
(which should be $\ket{01}-\ket{10}$). 
We are therefore motivated to study how physical 
states can be constructed from arbitrary product states.
\end{mot}

To specify the 
exchange statistics, we must label the particles, but in reality 
we can not do this for identical particles. 
Labelling is purely for the sake of narration. In fact, 
we will arrive at the so-called occupation number representation 
where labelling is impossible.

Before trying to construct physical states we first introduce 
some tools needed to define ``physical''.

\begin{dfn}[Permutation, exchange, parity]
    A \emph{permutation} is a 
    function 
    \begin{equation}
        P: \{(12\cdots N)\} \mapsto \{(12\cdots N)\}, 
    (123\cdots N) \mapsto (P1, P2, \ldots ,PN)
    \end{equation}
    where $\{(12\cdots N)\}$ denotes all possible $N$-bit strings 
    consisting of the integers 1,2, ..., $N$. 
    An \emph{exchange} is a permutation of the form 
    \begin{equation}
        P: (\ldots ,i, \ldots ,j, \ldots )
        \mapsto 
        (\ldots ,j, \ldots ,i, \ldots ),
    \end{equation}
    \textit{i.e.}~ $Pi=j, Pj=i$, and 
    any index other than $i,j$ is not affected.
    A permutation $P$ is defined to be \emph{even} if 
    it can be realized (\textit{i.e.}~implemented) by an even number of 
    exchanges, and \emph{odd} if it can be realized by 
    an odd number of exchanges. 
    \footnote{Formally proving that given a permutation 
    this realization is either even or odd 
    is outside the scope of this note.}
    We define $\abs{P} = 0$ for even permutations 
    and $\abs{P} = 1$ for odd ones; $\abs{P}$ is known as 
    the \textit{signature} or \textit{parity} of $P$. 
\end{dfn}

    For example, 
    $P: (123) \mapsto (213)$ is odd, 
    while $P': (123) \mapsto (231)$ is even.
    For any permutation $P$, $\abs{P} = \abs{P ^{-1}}$ 
    where $P ^{-1}$ is the inverse of the permutation.
    \footnote{Again, we don't formally prove this. Intuitively 
    this is natural. Any permutation can be implemented 
    via some steps of exchanges, and we can always invert a permutation by 
    doing all its steps in reverse, and obviously this gives the 
    same parity as the permutation itself.}

\begin{dfn}[Bosonic and Fermionic projectors]
    For an $N$-particle product state 
    \begin{equation}
        |\alpha_1, \alpha_2, \ldots ,\alpha_{N})
    \coloneqq \ket{\alpha_1} \otimes \ket{\alpha_2}\cdots \otimes \ket{\alpha_{N}}
    \end{equation}
    where $\forall i, \ket{\alpha_{i}} \in 
    \{\ket{\lambda_1}, \ldots ,\ket{\lambda_{d}}\}$ 
    and $\{\ket{\lambda_{j}}\}_{j=1}^{d}$ is an orthonormal basis 
    of the 1-body Hilbert space, 
    the Bosonic (Fermionic) projector $\mathcal{P}_{B}$ ($\mathcal{P}_{F}$) 
    are defined in terms of their action on product states,  
    \begin{equation}
        \mathcal{P}_{B}|\alpha_1, \ldots ,\alpha_{N})
        \coloneqq
        \frac{1}{N!} \sum_{P}\zeta^{\abs{P}} 
        |\alpha_{P1}, \ldots ,\alpha_{PN})
        \quad \text{with } \zeta = 1,
    \end{equation}
    \begin{equation}
        \mathcal{P}_{F}|\alpha_1, \ldots ,\alpha_{N})
        \coloneqq
        \frac{1}{N!} \sum_{P}\zeta^{\abs{P}} 
        |\alpha_{P1}, \ldots ,\alpha_{PN})
        \quad \text{with } \zeta = -1.
    \end{equation}
    When the distinction between Bosons and Fermions does not matter 
    we use $\mathcal{P}$ to denote $\mathcal{P}_{B}$ or $\mathcal{P}_{F}$. 
    The operators $\mathcal{P}_{B}, \mathcal{P}_{F}$ 
    are obviously linear.
\end{dfn}

Now we can define what we mean by ``physical'' using 
these projectors. 

\begin{dfn}[Bosonic and Fermionic states]
    \label{dfn:bosonic-and-fermionic-states}
    For any $N$-body state $\ket{\Psi}$, we can project it 
    onto the product basis 
    $\left\{ |\alpha_1, \ldots ,\alpha_{N}) \right\}$, 
    obtaining 
    $\ket{\Psi}=\sum_{\alpha_1, \ldots ,\alpha_{N}} 
    C_{\alpha_1, \ldots ,\alpha_{N}}
    |\alpha_1, \ldots ,\alpha_{N})$, 
    and then apply the projectors as previously defined. 
    We define that $\ket{\Psi}$ is Bosonic if it is preserved by $\mathcal{P}_{B}$ 
    and Fermionic if it is preserved by $\mathcal{P}_{F}$. 
    It follows that all Bosonic states form a subspace 
    $\mathcal{H}_{B}\coloneqq \left\{ \ket{\Psi} : \mathcal{P}_{B} \ket{\Psi}=\ket{\Psi} \right\}$
    and all Fermionic states form a subspace 
    $\mathcal{H}_{F}\coloneqq \left\{ \ket{\Psi} : \mathcal{P}_{F} \ket{\Psi}=\ket{\Psi} \right\}$ of the total product Hilbert space.
\end{dfn}

We have called $\mathcal{P}_{B}, \mathcal{P}_{F}$ projectors but are they? 
Let us check idempotence explicitly.

\begin{align*}
    \mathcal{P}^{2}|\alpha_{1}, \ldots ,\alpha_{N})
    =&\ 
    \frac{1}{N!}\sum_{P'}\zeta^{\abs{P'}}
    \frac{1}{N!}\sum_{P}\zeta^{\abs{P}}
    |\alpha_{P'P1}, \ldots ,\alpha_{P'PN})
    \\
    =&\ 
    \frac{1}{N!}\sum_{P'}
    \frac{1}{N!}\sum_{P}\zeta^{\abs{P'P}}
    |\alpha_{P'P1}, \ldots ,\alpha_{P'PN})
    \\
    =&\ 
    \frac{1}{N!}\sum_{P'}
    \frac{1}{N!}\sum_{P'P}
    \zeta^\abs{P'P}
    |\alpha_{P'P1}, \ldots ,\alpha_{P'PN})
    \\
    =&\ 
    \mathcal{P} |\alpha_1, \ldots ,\alpha_{N}),
\end{align*}
where in the second-to-last line the operator $\frac{1}{N!}\sum_{P'}$ 
equals to identity because it only involves superposition, not 
permutation. 
\footnote{If the relabelling of summation variables from $P',P$ to 
$P'P, P'$ seems strange, we can convince ourselves by noting that there 
exists a bijective mapping between $\left\{ (P',P) \right\}$ and 
$\left\{ (P'P, P) \right\}$. }

    The idenpotence of $\mathcal{P}$ means 
    $\mathcal{P} (\mathcal{P} \ket{\psi}) = \mathcal{P} \ket{\psi}$, \textit{i.e.}~for 
    any $\ket{\psi}$ the projected state $\mathcal{P} \ket{\psi}$ 
    belongs to the subspace $\{\ket{\Psi} : \mathcal{P} \ket{\Psi} = \ket{\Psi}\}$. 

\begin{graybox}
\begin{res}
    Given any many-body product state $|\alpha_1,\alpha_2, \ldots ,\alpha_{N})$, 
    $\mathcal{P}_{B}|\alpha_1,\alpha_2, \ldots ,\alpha_{N}) \in \mathcal{H}_{B}$, 
    and similarly 
    $\mathcal{P}_{F}|\alpha_1,\alpha_2, \ldots ,\alpha_{N}) \in \mathcal{H}_{F}$. 
    Physically, this means physical states are 
    constructed via ``scrambling'' product states with the 
    operator $\mathcal{P}$.
    This is known as the symmetrization of states.
\end{res}
\end{graybox}

\begin{rem}
    \label{rem:equivalence-to-commonly-known-dfn}
    The commonly known definition of a Bosonic (Fermionic) 
    state is that upon an exchange of particle index, 
    the state should be invariant (prepended with a minus sign). 
    This definiton is equivalent with ours, 
    as will be shown as follows. 

    First, we show that 
    \cref{dfn:bosonic-and-fermionic-states} 
    implies the commonly known property. 
    Denoting $\mathcal{P} |\alpha_1, \ldots ,\alpha_{N})$ with 
    $\ket{\alpha_1, \ldots ,\alpha_{N}}$, 
    we need to check that the state 
    after an exchange $P$ 
    \[
    \ket{\alpha_{P1}, \ldots ,\alpha_{PN}} 
    \coloneqq    
    \frac{1}{N!}
        \sum_{P'}
        \zeta^{\abs{P'}}
        |\alpha_{P'P1}, \ldots ,\alpha_{P'PN})
    \]
    equals to 
    $\zeta^{\abs{P}}\ket{\alpha_1, \ldots ,\alpha_{N}}$.
    Using the re-labelling trick, we have
    \begin{align*}
        \zeta^{\abs{P}}\ket{\alpha_1, \ldots ,\alpha_{N}}
        =&\ 
        \zeta^{\abs{P}}\frac{1}{N!}\sum_{P''}
        \zeta^{\abs{P''}}
        |\alpha_{P''1}, \ldots ,\alpha_{P''N}) \\
        =&\ 
        \zeta^{\abs{P}}\frac{1}{N!}\sum_{P'P}\zeta^{\abs{P'P}}
        |\alpha_{P'P1}, \ldots ,\alpha_{P'PN}) \\
        =&\ 
        \zeta^{\abs{P}}\frac{1}{N!}\sum_{P'P}
        \zeta^{\abs{P'}}\zeta^{\abs{P}}
        |\alpha_{P'P1}, \ldots ,\alpha_{P'P N}) \\
        =&\ 
        \frac{1}{N!}
        \sum_{P'}\zeta^{\abs{P'}}
        |\alpha_{P'P1}, \ldots ,\alpha_{P'P N}),
    \end{align*}
    where when re-labelling $P''$ to $P'P$ we 
    fix $P$ to be the exchange at concern and 
    $P'$ is to be summed over.

    Then, we show that the commonly known definition 
    implies our definition. If an exchange $E$ always 
    leads to an extra prefactor $\zeta$, \textit{i.e.}
    \[
        \ket{\alpha_{E 1}, \ldots ,\alpha_{EN}}
        =
        \zeta \ket{\alpha_1, \ldots ,\alpha_{N}},
    \]
    then a permutation 
    that can be implemented via $M$ exchanges
    \[
        P = E_{M} \circ \cdots \circ E_1
    \]
    would take 
    $\ket{\alpha_{1}, \ldots ,\alpha_{N}}$ to 
    \[
        \zeta^{M} \ket{\alpha_{P 1}, \ldots ,\alpha_{PN}}
        =
        \zeta^{\abs{P}}
        \ket{\alpha_{P 1}, \ldots ,\alpha_{PN}},
    \]
    where we have used the definition of $\abs{P}$. 
    It then follows that $\ket{\alpha_1, \ldots ,\alpha_{N}}$ 
    is preserved by $\mathcal{P}$.
\end{rem}

\begin{rem}
    Consistency requires that no state can be simultaneously 
    Bosonic and Fermionic. 
    Formally this manifests in the fact that 
    $\mathcal{P}_{B}$ maps 
    all Fermionic states to the null vector and hence 
    $\mathcal{H}_{B} \bot \mathcal{H}_{F}$. 
    Let us check this explicitly: 
    the state
    $\mathcal{P}_{B} \mathcal{P}_{F}|\alpha_1, \ldots ,\alpha_{N})$ 
    equals to
    \begin{align*}
        &\ \frac{1}{N!}\sum_{P'}(+1)^{\abs{P'}}
        \frac{1}{N!}\sum_{P}(-1)^{\abs{P}}
        |\alpha_{P'P1}, \ldots ,\alpha_{P'PN}) \\
        =&\ 
        \frac{1}{N!}\sum_{P'}(-1)^{\abs{P'}}
        \frac{1}{N!}\sum_{P'P}(-1)^{\abs{P'P}}
        |\alpha_{P'P1}, \ldots ,\alpha_{P'PN})
        \\
        =&\ 
        \frac{1}{N!}
        \left( \sum_{\text{odd }P'}+\sum_{\text{even }P'} \right) 
        (-1)^{\abs{P'}}
        \mathcal{P}_{F} |\alpha_1, \ldots ,\alpha_{N})
        \\
        =&\ 
        0,
    \end{align*}
    where the first equality holds because 
    whatever the parity of $P'$, $(-1)^{2\abs{P'}}$ is always 1, and 
    the third equality holds because 
    for each even $P'$ there is an odd one that cancels its contribution. 

    In fact, using the commonly known definition 
    as mentioned in \cref{rem:equivalence-to-commonly-known-dfn} 
    to show this is easier. We take a generic Bosonic state 
    $\ket{\beta_1, \ldots ,\beta_{N}}$ and a generic Fermionic state 
    $\ket{\phi_{1}, \ldots ,\phi_{N}}$, their inner product 
    is invariant upon a simultaneous permutation 
    because they are both superpositions of product states: 
    \[
        \bra{\beta_1, \ldots ,\beta_{N}} 
        \ket{\phi_1, \ldots ,\phi_{N}}
        =
        \bra{\beta_{P 1}}, \ldots ,\beta_{PN} 
        \ket{\phi_{P 1}, \ldots ,\phi_{PN}}.
    \]
    Setting $P$ to be an exchange 
    \[
        P = (1,2, \ldots ,N) \mapsto 
        (2,1, \ldots ,N)
    \]
    gives 
    \begin{align*}
        \bra{\beta_1, \ldots ,\beta_{N}} 
        \ket{\phi_1, \ldots ,\phi_{N}}
        =&\ 
        \bra{\beta_2,\beta_1, \ldots }
        \ket{\phi_2,\phi_1, \ldots } \\
        =&\ 
        - \bra{\beta_1,\beta_2, \ldots }
        \ket{\phi_1, \phi_2, \ldots },
    \end{align*}
    so the inner product equals to its 
    opposite number and thus vanishes.
\end{rem}

\section{The orthonormal Fock basis}
\begin{mot}
    We would like to construct orthonormal bases within 
    $\mathcal{H}_{B}$ and $\mathcal{H}_{F}$. 
    Having taken care of exchange statistics, we now 
    normalize the symmetrized state. After that 
    we shall study whether the normalized states are orthogonal. 
\end{mot}

    We calculate $\norm{\mathcal{P}|\alpha_1, \ldots ,\alpha_{N})}$ 
    by taking the inner product 
    $(\alpha_1, \ldots ,\alpha_{N}|\mathcal{P}^{\dagger}\mathcal{P}|\alpha_1, \ldots ,\alpha_{N})$ and taking its square root. 
    Here it would be ideal if $\mathcal{P}$ was 
    Hermitian, because in that case we can use 
    idempotence and reduce the inner product to 
    \[
    (\alpha_1, \ldots ,\alpha_{N}|\mathcal{P}|\alpha_1, \ldots ,\alpha_{N}),
    \]
    which is easy to calculate thanks to the 
    orthogonality of the product basis.
    
    We establish Hermiticity by verifying that $\mathcal{P}$ meets 
    the Hermiticity condition: for any product states 
    $|\alpha_{1}, \ldots ,\alpha_{N})$ and 
    $|\alpha'_{1}, \ldots ,\alpha'_{N})$,
    \begin{equation}
        (\alpha_1, \ldots ,\alpha_{N}|\mathcal{P}|
        \alpha'_{1}, \ldots ,\alpha'_{N})
        =
        (\alpha'_1, \ldots ,\alpha'_{N}|\mathcal{P}|
        \alpha_{1}, \ldots ,\alpha_{N})^{*},
    \label{eqn:hermiticity-condition-for-projector}
    \end{equation}
    which is equivalent to 
    \[
        \sum_{P}\zeta^{\abs{P}}
        (\alpha_1, \ldots ,\alpha_{N}|
        \alpha'_{P1}, \ldots ,\alpha'_{PN})
        =
        \sum_{P'}\zeta^{\abs{P'}}
        (\alpha_{P'1}, \ldots ,\alpha_{P'N}|
        \alpha'_{1}, \ldots ,\alpha'_{N}).
    \]
    Each LHS term 
    \[
        \zeta^{\abs{P}}
    (\alpha_1, \ldots ,\alpha_{N}|
    \alpha'_{P1}, \ldots ,\alpha'_{PN})
    \]
    can be equated (via applying $P ^{-1}$ on 
    the ``bra'' and the ``ket'' simultaneously) with the RHS term 
    \[
        \zeta^{\abs{P ^{-1}}}
    (\alpha_{P ^{-1} 1}, \ldots ,\alpha_{P ^{-1} N}|
    \alpha'_{1}, \ldots ,\alpha'_{N}),
    \]
    where $\zeta^{\abs{P}}=\zeta^{\abs{P^{-1}}}$ and 
    the inner product between 
    two product states is always preserved 
    by a simultaneous permutation, 
    which proves \cref{eqn:hermiticity-condition-for-projector}.

    Continuing with our task of normalization, 
    \begin{align*}
        &\ \norm{\mathcal{P} |\alpha_1, \ldots ,\alpha_{N})}^{2} \\
        =&\ 
        (\alpha_1, \ldots ,\alpha_{N}|
        \sum_{P}\zeta^{\abs{P}} \frac{1}{N!}
        |\alpha_{P1}, \ldots ,\alpha_{PN}) \\
        =&\ 
        \frac{1}{N!} \prod_{n_{\lambda_{i}} \neq 0} n_{\lambda_{i}} ! 
        \quad \text{for Bosons, and } 
        \frac{1}{N!} \quad \text{for Fermions},
    \end{align*}
    where in the last equality, 
    contributing permutations are determined 
    by the orthogonality of the product basis, and 
    in particular the only contributing permutation 
    in the Fermion case is the identity. 
    The condition $n_{\lambda_{i}}\neq 0$ in the product 
    is imposed 
    so that the product does not vanish. 

    \begin{graybox}
    \begin{res}
        After normalization the vector 
        $\mathcal{P}|\alpha_1, \ldots ,\alpha_{N})$
        becomes 
        \[
        \frac{\sqrt{N!}}{\sqrt{\prod_{n_{\lambda_{i}}\neq 0}n_{\lambda_{i}}!}}
        \mathcal{P}|\alpha_1, \ldots ,\alpha_{N}).
        \]
    \label{res:normalization-of-scrambled-states}
    \end{res}
    \end{graybox}

    Motivated by this result we define: 

    \begin{dfn}
        The normalized and symmetrized state 
        is defined as
        \begin{equation}
            \ket{\alpha_1, \ldots ,\alpha_{N}}
            \coloneqq
        \frac{\sqrt{N!}}{\sqrt{\prod_{n_{\lambda_{i}}\neq 0}n_{\lambda_{i}}!}}
        \mathcal{P}|\alpha_1, \ldots ,\alpha_{N})
        =
        \frac{\sqrt{N!}}{\sqrt{\prod_{n_{\lambda_{i}}\neq 0}n_{\lambda_{i}}!}}
        \frac{1}{N!} \sum_{P}\zeta^{\abs{P}}
        |\alpha_1, \ldots ,\alpha_{N})
        \end{equation}
        and the ``partially normalized'' state 
        \footnote{
        The ``partially normalized'' state is useful in some 
        miscellaneous situations, but it is not our primary focus, since 
        we have not finished the task of finding an orthonormal basis. 
        }
        is defined as
        \begin{equation}
            |\alpha_1, \ldots ,\alpha_{N}\}
            \coloneqq
            \sqrt{N!} \mathcal{P} |\alpha_1, \ldots ,\alpha_{N}).
        \end{equation}
        Note that the partially normalized state and the normalized state 
        are only different in the Bosonic case, because they only 
        differ in the prefactor $\left( \prod n_{\lambda_{i}}! \right)^{1 /2}$. 
        \label{dfn:normalized-states-and-partially-normalized-states}
    \end{dfn}

    Having obtained normalized states, 
    we go on to study their orthogonality. 
    For this purpose we note that 
    the only information carried by $\ket{\alpha_1, \ldots ,\alpha_{N}}$ 
    is 
    \begin{enumerate}
        \item 
        Which 1-body states among $\{\ket{\lambda_{i}}\}_{i = 1}^{d}$ 
        are occupied, and 
        \item
        for each occupied 1-body state, 
        how many particles occupy it.
    \end{enumerate}

    We can therefore encode these information 
    in the following notation, 

    \begin{dfn}[Occupation number representation]
        For the $N$-body normalized state 
        $\ket{\alpha_1, \ldots ,\alpha_{N}}$, 
        the \textit{occupation number representation} 
        (also known as the Fock state) is 
        $\ket{n_{\lambda_{1}}, \ldots ,n_{\lambda_{d}}}$ 
        where $n_{\lambda_{j}}$ is the number of particles 
        in the state $\ket{\lambda_{j}}$ and 
        $\sum_{j=1}^{d}n_{\lambda_{j}}=N$.
    \end{dfn}

    Immediately we see that the Fock state 
    carries the two pieces of information we mentioned. 
    The unoccupied 1-body states have occupation number zero. 
    Besides, it is impossible to label particles 
    in the occupation number representation.

    With this compact notation we study orthogonality. 
    Intuitively we would expect 
    $n_{\lambda_{1}}\neq n'_{\lambda_{1}}$ to 
    imply 
    \begin{equation}
        \bra{n_{\lambda_{1}}, \ldots }\ket{n'_{\lambda_{1}}, \ldots }=0,
    \label{eqn:orthogonality-of-Fock-states}
    \end{equation}
    since the occupation number of $\ket{\lambda_{1}}$ 
    \textit{completely distinguishes between} these states and 
    consequently their overlap should vanish.
    \footnote{
        More concretely, if there exists a Hermitian operator 
        $\hat{n}_{\lambda_{1}}$ which extracts the occupation number 
        of $\ket{\lambda_{1}}$, then these two states belong to 
        different eigenspaces of $\hat{n}_{\lambda_{1}}$, so they 
        must be orthogonal. We will see 
        in \cref{rem:from-Fock-states-to-number-operator} 
        that there indeed 
        exists such an operator.
    }

    Now let us verify orthogonality explicitly. 
    Denote $\sum_{\lambda_{i}}n_{\lambda_{i}}$ with $N$, and
    $\sum_{\lambda_{i}}n'_{\lambda_{i}}$ with $M$.

    If $N=M$, we illustrate with an example; 
    generalization is trivial. Say 
    $n_{\lambda_{1}}=2$ and $n'_{\lambda_{1}}=1$. 
    In the inner product $\bra{n_{\lambda_{1}}, \ldots }\ket{n'_{\lambda_{1}}, \ldots }$, each side is decomposed into a sum over product states. 
    Whatever the permutation, $\ket{\lambda_1}$ always appear 
    twice in the LHS and once in the RHS, so there is at least 
    one $\bra{\lambda_{1}}$ in the LHS which is taken inner product with 
    a 1-body state that is not $\ket{\lambda_{1}}$, hence 
    the inner product vanishes. 
    This analysis extends to every term in the sum, 
    so the total inner product vanishes.

    If $N \neq M$ (say $N<M$), then we have to first consider 
    what exactly we mean by the inner product between 
    an $N$-body product state and an $M$-body product state. 
    By the definition of inner product, 
    we have to write the $N$-body product state 
    as an $M$-body product state, and the only sensible way 
    to do this is to tensor it with $(M-N)$ 1-body null vectors, 
    since this is the only way to guarantee nothing happens 
    in the $(M-N)$ 1-body Hilbert spaces. 
    Given this assumption, 
    \cref{eqn:orthogonality-of-Fock-states} follows. 
    \footnote{
        We will see that the orthogonality of states 
        with different total particle numbers 
        is also required by the definition of 
        creation and annihilation operators. 
        However, in arguing in terms of product states here, 
        we get a more intuitive understanding. 
    }


    \begin{graybox}
    \begin{res}
        Normalizing the symmetrized states and 
        switching to the occupation number representation 
        gives an orthonormal basis in the physical Hilbert spaces, 
        known as the Fock basis. A Fock state is 
        $\ket{n_{\lambda_{1}}, \ldots ,n_{\lambda_{d}}}$ 
        where $n_{\lambda_{j}}$ is the number of particles 
        in the state $\ket{\lambda_{j}}$ and 
        $\sum_{j=1}^{d}n_{\lambda_{j}}=N$.
    \end{res}
    \end{graybox}

    \begin{rem}
        \label{rem:from-Fock-states-to-number-operator}
        Because the Fock states are orthonormal, 
        we can (but might choose not to) 
        \textit{define} the number operator 
        in terms of how it acts on Fock states: 
        \[
            \hat{n}_{\lambda_{i}}
            \ket{\cdots ,n_{\lambda_{i}}, \ldots }
            =
            n_{\lambda_{i}}
            \ket{\cdots ,n_{\lambda_{i}}, \ldots }.
        \]
        It follows that 
        \[
            \bra{\cdots ,n_{\lambda_{i}}, \ldots }
            \hat{n}^{\dagger}_{\lambda_{i}}
            =
            n^{*}_{\lambda_{i}}
            \bra{\cdots ,n_{\lambda_{i}}, \ldots }
            =
            n_{\lambda_{i}}
            \bra{\cdots ,n_{\lambda_{i}}, \ldots }
        \]
        and the Hermitian condition 
        \[
            \bra{\cdots ,n_{\lambda_{i}}, \ldots }
            \hat{n}_{\lambda_{i}}
            \ket{\cdots ,n'_{\lambda_{i}}, \ldots }
            =
            \bra{\cdots ,n'_{\lambda_{i}}, \ldots }
            \hat{n}_{\lambda_{i}}
            \ket{\cdots ,n_{\lambda_{i}}, \ldots }^{*}
        \]
        is met because, thanks to the 
        orthogonality of the Fock basis, both sides 
        in this equation are 
        $n_{\lambda_{i}} \delta_{n_{\lambda_{i}},n'_{\lambda_{i}}} $.
    \end{rem}

    \begin{rem}
        In the preceding discussion, 
        all Fock states with total particle number 
        $N\ge 1$ can be constructed 
        via symmetrizing and normalizing 
        product states. 
        However, the Fock state $\ket{0}$ 
        with total particle number zero 
        seems difficult to make sense of, 
        since we can not project it onto 
        product states. 
        We will gradually see the 
        physical picture of this state 
        in the next section.
    \end{rem}

    \begin{rem}
        Different Fock states 
        can correspond to the same set of 
        occupation numbers. 
        For example, 
        consider two Fermionic states 
        with 1-body space 
        $\{\ket{\alpha},\ket{\beta}\}$: 
        $\ket{\alpha \beta} \propto |\alpha \beta) - |\beta \alpha)$ 
        and 
        $\ket{\beta \alpha} \propto |\beta \alpha) - |\alpha \beta)$. 
        They correspond to the same 
        occupation numbers 
        $n_{\alpha}=1, n_{\beta}=1$, 
        but differ by an overall phase $-1$. 
        In practice we always choose a ordering 
        convention of writing down 1-body states 
        so that this overall phase does not mess up 
        our calculations. 
        For example, 
        to get rid of the $-1$ phase mentioned above, 
        we can choose to always write 
        $\ket{\alpha}$ before $\ket{\beta}$ 
        in a Fock state. 
        More generally, we might choose to always 
        write down low-energy 1-body states 
        before high-energy ones. 
    \label{rem:partially-normalized-states-can-differ-by-overall-sign}
    \end{rem}

\section{Creation and annihilation}

\begin{mot}
    Previously we were motivated 
    by the need to take care of 
    exchange statistics of quantum states. 
    But there is more to second quantization. 
    Under the Fock basis, 
    quantum states are labelled by 
    their occupation numbers, \textit{i.e.}~the 
    occupation numbers become 
    degrees of freedom. 
    We are thus motivated to 
    study processes that manipulate 
    these degrees of freedom. 
    We will construct operators that 
    do this and study their properties.
\end{mot}

\subsection{Creation, annihilation and number operators}

\begin{dfn}[Creation operator]
    The \emph{creation operator} 
    $\hat{\psi}_{\alpha}$ 
    ``creates'' one particle 
    in the 1-body state $\ket{\alpha}$. 
    To be specific, 
    it is defined 
    by its action on a partially normalized state, 
    \begin{equation}
        \hat{\psi}_{\alpha}
        |\alpha_1, \ldots ,\alpha_{N}\}
        =
        |\alpha \alpha_{1}, \ldots ,\alpha_{N}\}
    \end{equation}
    with 
    $\alpha, \alpha_1, \ldots ,\alpha_{N}$ 
    not necessarily different. 
    Note that $\alpha$ is written 
    on the left of the state 
    $|\alpha\alpha_1, \ldots ,\alpha_{N}\}$; 
    recalling \cref{rem:partially-normalized-states-can-differ-by-overall-sign}, 
    this convention should be obeyed 
    consistently to avoid confusion.
\end{dfn}

Fock states are related to partially normalized states 
as 

\[
    |\alpha \alpha_{1}, \ldots ,\alpha_{N}\}
    =
    \sqrt{
        \prod_{\alpha_{i}\neq \alpha}
        n_{\alpha_{i}}!
    }
    \sqrt{(n_{\alpha}+1)!}
    \ket{\alpha \alpha_1, \ldots ,\alpha_{N}},
\]

\[
    |\alpha_1, \ldots ,\alpha_{N}\}
    =
    \sqrt{
        \prod_{\alpha_{i}}
        n_{\alpha_{i}}!
    }
    \ket{\alpha_1, \ldots ,\alpha_{N}},
\]
hence the creation operator 
acts on Fock states as 
\begin{align*}
    \hat{\psi}^{\dagger}_{\alpha}
    \ket{\alpha_1, \ldots ,\alpha_{N}}
    =&\ 
    \hat{\psi}^{\dagger}_{\alpha}
    \frac{1}{\sqrt{
        \prod_{\alpha_{i}}
        n_{\alpha_{i}}
    }}
    |\alpha_1, \ldots ,\alpha_{N}\} \\
    =&\ 
    \frac{1}{\sqrt{
        \prod_{\alpha_i}
        n_{\alpha_i}
    }}
    |\alpha \alpha_{1},\cdots, \alpha_{N}\} \\
    =&\ 
    \sqrt{n_{\alpha}+1}
    \ket{\alpha \alpha_{1}, \ldots ,\alpha_{N}}.
\end{align*}

\begin{mot}
    $\hat{\psi}^{\dagger}_{\alpha}$
    maps an $N$-body state to an $(N+1)$-body 
    state, \textit{i.e.}~it looks like 
    \[
      \sum_{N}\ket{(N+1)\text{-body state}}\bra{N\text{-body state}}.
    \]
    We therefore expect 
    $\hat{\psi}_{\alpha}$, its Hermitian adjoint, 
    to map an $(N+1)$-body state 
    to an $N$-body state, 
    which can be interpreted as the 
    ``annihilation'' of a particle. 
    Let us study the properties of 
    $\hat{\psi}_{\alpha}$ 
    to check our intuition.
\end{mot}

We project 
$\hat{\psi}_{\alpha} |\alpha'_{1}, \ldots ,\alpha'_{m}\}$
onto $\{\alpha_1, \ldots ,\alpha_{n}|$ to 
study a matrix element of $\hat{\psi}_{\alpha}$. 

\begin{align*}
    \{ \alpha_1, \ldots ,\alpha_{n}
    |\hat{\psi}_{\alpha}|
    \alpha'_{1}, \ldots ,\alpha'_{m}\}
    =&\ 
    \{ \alpha'_{1}, \ldots ,\alpha'_{m} |
    \hat{\psi}^{\dagger}_{\alpha}|
    \alpha_1, \ldots ,\alpha_{n}\}
    ^{*}
    \\
    =&\ 
    \{\alpha'_{1}, \ldots ,\alpha'_{m}
    |
    \alpha \alpha_{1}, \ldots ,\alpha_{n}\}
    ^{*}.
\end{align*}

Therefore the only nonvanishing matrix elements 
are those where $m=n+1$.

Now we project 
$\hat{\psi}_{\alpha}|\alpha'_{1}, \ldots ,\alpha'_{m}\}$ 
onto a complete basis $\left\{ |\alpha_1, \ldots ,\alpha_n\} \right\}$ 
where $n=m-1$ 
to obtain the formula of 
$\hat{\psi}_{\alpha}|\alpha'_{1}, \ldots ,\alpha'_{m}\}$.
For this purpose we need 
the identity within the $n$-body space 
\[
    \mathds{1}_{n}
    =
    \sum_{\alpha_1, \ldots ,\alpha_{n}}
    |\alpha_1, \ldots ,\alpha_{n})(\alpha_1, \ldots ,\alpha_{n}|, 
\]
which becomes 
\[
    \mathds{1}_{n}
    =
    \sum_{\alpha_1, \ldots ,\alpha_{n}}
    \frac{1}{n!}
    |\alpha_1, \ldots ,\alpha_{n}\}
    \{\alpha_1, \ldots ,\alpha_{n}|
\]
when expressed in terms of the basis 
$\{|\alpha_1, \ldots ,\alpha_{n}\}\}$.

Proceeding with our expansion, 

\begin{align*}
    \hat{\psi}_{\alpha}
    |\alpha'_{1}, \ldots ,\alpha'_{m}\}
    =&\ 
    \sum_{\alpha_1, \ldots ,\alpha_{n}}
    \frac{1}{n!}
    \{\alpha_1, \ldots ,\alpha_{n}|
    \hat{\psi}_{\alpha}|
    \alpha'_{1}, \ldots ,\alpha'_{m}\}
    \cross 
    |\alpha_1, \ldots ,\alpha_{n}\} \\
    =&\ 
    \sum_{\alpha_1, \ldots ,\alpha_{m-1}}
    \frac{1}{(m-1)!}
    \sqrt{m!}
    (\alpha \alpha_1, \ldots ,\alpha_n
    |\frac{1}{m!}\sum_{P}\zeta^{\abs{P}}|
    \alpha'_{P1}, \ldots ,\alpha'_{Pm})
    \sqrt{m!}
    \cross 
    |\alpha_1, \ldots ,\alpha_{m-1}\}
\end{align*}
where the permutation $P$ involves $m$ indices, 
while the vector 
$|\alpha_1, \ldots ,\alpha_{m-1}\}$ 
involves $m-1$ indices. The orthogonality of 
the product basis gives 
\begin{align*}
    \hat{\psi}_{\alpha}
    |\alpha'_{1}, \ldots ,\alpha'_{m}\}
    =
    \frac{1}{(m-1)!}
    \sum_{P}\zeta^{\abs{P}}
    \delta_{\alpha,\alpha'_{P1}}
    |\alpha'_{P2}, \ldots ,\alpha'_{Pm}\}.
\end{align*}
But we want to express our result in a way 
not involving permuted indices, 
so we carry out the following permutation
\footnote{
    When permuting the indices in the vestor, 
    we are just 
    re-labelling the summation index $P$.
}
for the state 
$|\alpha'_{P2}, \ldots ,\alpha_{Pm}\}$ 
:
\[
    (P1, P2,\ldots ,Pm)
    \mapsto
    (P1,1,2,3, \ldots ,
    P1-1,P1+1, \ldots ,m),
\]
which is $P^{-1}$ followed by 
$(P1-1)$ nearest-neighbor transpositions and has 
therefore the signature $\abs{P ^{-1}}+P1-1$. 
After this permutation, the $\zeta$ prefactor becomes 
$\zeta^{\abs{P}+\abs{P ^{-1}} + P1 -1} = \zeta^{P1-1}$, 
resulting in
\begin{align*}
    \hat{\psi}_{\alpha}
    |\alpha'_{1}, \ldots ,\alpha'_{m}\}
    =&\ 
    \frac{1}{(m-1)!}
    \sum_{P}
    \zeta^{P1-1}
    \delta_{\alpha,\alpha'_{P1}}
    |\alpha'_{1}, \ldots ,
    \alpha'_{P1-1},\alpha'_{P1+1}, \ldots ,
    \alpha'_{Pm}\} \\
    =&\ 
    \left(\sum_{P1=1}+
    \sum_{P1=2}+ \cdots +
    \sum_{P1=m}\right) 
    \frac{1}{(m-1)!}
    \zeta^{P1-1}
    \delta_{\alpha,\alpha'_{P1}}
    |\alpha'_{1}, \ldots ,
    \alpha'_{P1-1},\alpha'_{P1+1}, \ldots ,
    \alpha'_{Pm}\} \\
    =&\ 
    \sum_{j=1}^{m}
    \zeta^{j-1}
    \delta_{\alpha,\alpha'_{j}}
    \left|\alpha'_{1}, \ldots ,\alpha'_{j-1},
    \alpha'_{j+1}, \ldots ,\alpha'_{m}\right\},
\end{align*}
where the last equation holds because 
the summand 
in the second-to-last line only depends on $P1$, 
so for each summation in the bracket 
$
    \left(\sum_{P1=1}+
    \sum_{P1=2}+ \cdots +
    \sum_{P1=m}\right) 
$,
there are $(m-1)!$ identical terms, which 
cancels the prefactor $1 /(m-1)!$.

This equation says to: 1) go to the index $j$ 
in the array 
$(\alpha'_{1}, \ldots ,\alpha'_{m})$ 
and check whether $\alpha'_{j}=\alpha$, 
2) if yes, register a contributing term 
$
    \zeta^{j-1}
    |\alpha'_{1}, \ldots ,\alpha'_{j-1},
    \alpha'_{j+1}, \ldots ,\alpha'_{m}\}
$, 
3) if no, don't register any contribution, and 
4) sum up the contributing terms.
Therefore, the sum 
$\sum_{j=1}^{m}$ picks up $n_{\alpha}$ 
contributing terms, all of which 
having one less particle in the 1-body state 
$\ket{\alpha}$ than the state 
$|\alpha'_{1}, \ldots ,\alpha'_{m}\}$.

In terms of Fock states, 

\[
    \hat{\psi}_{\alpha}
    \ket{\alpha'_{1}, \ldots ,\alpha'_{m}}
    \cross 
    \sqrt{
        \prod_{\alpha'\neq \alpha}
        n_{\alpha'}
    }
    \sqrt{n_{\alpha}!}
    =
    \sum_{j=1}^{m}
    \zeta^{j-1}
    \delta_{\alpha,\alpha'_{j}}
    \ket{
        \alpha'_{1}, \ldots ,
        \alpha'_{j-1},\alpha'_{j+1},
        \cdots ,\alpha'_{m}
    }
    \cross 
    \sqrt{
        \prod_{\alpha'\neq \alpha}
        n_{\alpha'}
    }
    \sqrt{(n_{\alpha}-1)!},
\]
hence 
\[
    \hat{\psi}_{\alpha}
    \ket{\alpha'_{1}, \ldots ,\alpha'_{m}}
    =
    \begin{cases}
        \ket{\text{null}} & \text{if  } n_{\alpha}=0,\\
        n_{\alpha}^{-1 /2}
        \sum_{j=1}^{m}
    \zeta^{j-1}
    \delta_{\alpha,\alpha'_{j}}
    \ket{
        \alpha'_{1}, \ldots ,
        \alpha'_{j-1},\alpha'_{j+1},
        \cdots ,\alpha'_{m}
    }
    & \text{if  } n_{\alpha}\neq 0,
    \end{cases}
\]
where for the case $n_{\alpha}=1$, 
the factor $\sqrt{(n_{\alpha}-1)!}$ 
is not multiplied, because such factors 
are only considered when the occupation number 
is not zero, see \cref{res:normalization-of-scrambled-states} 
and \cref{dfn:normalized-states-and-partially-normalized-states}. 
This is also physical: having occupation number zero 
for some 1-body state does not mean the state is $\ket{\text{null}}$. 
For the case $n_{\alpha}=0$, 
$\delta_{\alpha,\alpha'_{j}}$ is always zero, 
so the resulting state is $\ket{\text{null}}$.
Physically, we 
\emph{should not} be able to annihilate a particle 
in the state $\ket{\alpha}$ when there is nothing 
to be annihilated in the state $\ket{\alpha}$. 
That the RHS is $\ket{\text{null}}$ 
prohibits this absurd annihilation process, because 
anything that follows would 
be happening to $\ket{\text{null}}$ and therefore 
have no physical 
consequence. 
\footnote{
For all operators, 
$\ket{\text{null}}$ will be mapped to itself, 
and the expectation value under $\ket{\text{null}}$ 
is always zero.
}

Although its expression seems complicated, 
the norm of 
$\hat{\psi}_{\alpha}\ket{\alpha'_{1}, \ldots ,\alpha'_{m}}$ 
is always $\sqrt{n_{\alpha}}$, because:
If $n_{\alpha}=0$ then this quantity equals to zero, 
our claim is true. Otherwise, for bosons 
the $n_{\alpha}$ contributing terms are all identical, 
so the prefactor is 
\[
    \frac{1}{\sqrt{n_{\alpha}}}
    \cross 
    n_{\alpha}
    =
    \sqrt{n_{\alpha}}.
\]
For fermions $n_{\alpha}$ is 1 if not zero, 
so there is only one contributing term and the norm 
is simply 
\[
    \frac{1}{\sqrt{1}}\cross 1=1,
\]
with the overall phase $\zeta^{j-1}$ 
which doesn't affect the vector's norm.


\begin{graybox}

\begin{res}
    The operator $\hat{\psi}_{\alpha}$
    maps the state $|\alpha'_{1}, \ldots ,\alpha'_{m}\}$ to
    \[
        \sum_{j=1}^{m}
        \zeta^{j-1}
    \delta_{\alpha,\alpha'_{j}}
    |\alpha'_{1}, \ldots ,\alpha'_{j-1},
    \alpha'_{j+1}, \ldots ,\alpha'_{m}\},
    \]
    which can be interpreted as 
    the annihilation of a particle 
    in the 1-body state $\ket{\alpha}$. 
    For Fock states,
    \begin{equation}
    \hat{\psi}_{\alpha}
    \ket{\alpha'_{1}, \ldots ,\alpha'_{m}}
    =
    \begin{cases}
        \ket{\text{null}} & \text{if  } n_{\alpha}=0,\\
        n_{\alpha}^{-1 /2}
        \sum_{j=1}^{m}
    \zeta^{j-1}
    \delta_{\alpha,\alpha'_{j}}
    \ket{
        \alpha'_{1}, \ldots ,
        \alpha'_{j-1},\alpha'_{j+1},
        \cdots ,\alpha'_{m}
    }
    & \text{if  } n_{\alpha}\neq 0.
    \end{cases}
    \label{eqn:annihilation-operator-on-fock-state}
    \end{equation}
    and 
    \begin{equation}
        \norm{
            \hat{\psi}_{\alpha}
            \ket{\alpha'_{1}, \ldots ,\alpha'_{m}}
        }
        =
        \sqrt{n_{\alpha}}.
    \end{equation}
\end{res}    

\end{graybox}

\begin{rem}
    The prefactor $\zeta^{j-1}$ in 
    \cref{eqn:annihilation-operator-on-fock-state} 
    seems strange because 
    it depends on the specific labelling of particles, 
    while we have claimed Fock states and their 
    properties to 
    depend exclusively on occupation numbers. 
    In fact this prefactor has very simple 
    consequences. 
    For bosons it is always 1, which is trivial. 
    For fermions there can be at most 
    one contributing term, 
    so this prefactor is an overall phase. 
    Recalling \cref{rem:partially-normalized-states-can-differ-by-overall-sign}, 
    we see that this overall phase arises from our 
    choice of ordering convention of 1-body states 
    when writing down 
    $|\alpha'_{1}, \ldots ,\alpha'_{m}\}$.
\end{rem}

\begin{rem}
    Here we have defined 
    the creation and annihilation operators 
    in terms of partially normalized states 
    and \emph{derived} their action 
    on Fock states. There is another approach
    \cite[][\S\ 2.1]{altland_condensed_2023}
    which 
    defines these operators in terms of 
    how they act on Fock states. 
    Had we used this approach, 
    it would be hard to see how 
    $\hat{\psi}_{\alpha}$ maps 
    a quantum state where $n_{\alpha}=0$ 
    to the null vector, 
    but in our approach this is a 
    natural consequence.
\end{rem}

\begin{mot}
    After creating and annihilating 
    a particle in the 1-body state $\ket{\alpha}$, 
    the occupation number $n_{\alpha}$ 
    remains the same. Besides, 
    \[
        \norm{
        \hat{\psi}^{\dagger}_{\alpha}
        \ket{\alpha'_{1}, \ldots ,\alpha'_{N}}
        }
        =
        \sqrt{n_{\alpha}+1},\quad
        \norm{
        \hat{\psi}_{\alpha}
        \ket{\alpha'_{1}, \ldots ,\alpha'_{N}}
        }
        =
        \sqrt{n_{\alpha}},
    \]
    so applying the creation and annihilation 
    operators 
    not only manipulates occupation numbers, 
    but also adds overall scale factors 
    $\sqrt{n_{\alpha}+1}$, $\sqrt{n_{\alpha}}$. 
    We are therefore motivated 
    to use these scale factors 
    to construct the number operator 
    as mentioned in \cref{rem:from-Fock-states-to-number-operator}.
\end{mot}

We want to obtain $n_{\alpha}$ by 
$\sqrt{n_{\alpha}}\cross \sqrt{n_{\alpha}}$, 
but $\hat{\psi}^{\dagger}_{\alpha}$ 
is associated with $\sqrt{n_{\alpha}+1}$. 
Hence we annihilate an operator first so that 
the scale factor associated with $\hat{\psi}^{\dagger}_{\alpha}$ 
becomes $\sqrt{n_{\alpha}}$; \textit{i.e.}\ we try 
the operator $\hat{\psi}^{\dagger}\hat{\psi}$.

If $n_{\alpha}\neq 0$, 
\begin{align*}
    \hat{\psi}^{\dagger}\hat{\psi}
    \ket{\alpha'_{1}, \ldots ,\alpha'_{N}}
    =&\ 
    \hat{\psi}^{\dagger}
    \frac{1}{\sqrt{n_{\alpha}}}
    \sum_{j=1}^{d}
    \zeta^{j-1}\delta_{\alpha,\alpha'_{j}}
    \ket{\alpha'_{1}, \ldots ,\alpha'_{j-1},
    \alpha'_{j+1}, \ldots ,\alpha'_{N}} \\
    =&\ 
    \frac{\sqrt{(n_{\alpha}-1)+1}}{\sqrt{n_{\alpha}}}
    \sum_{j=1}^{d}
    \zeta^{j-1}\delta_{\alpha,\alpha'_{j}}
    \ket{\alpha \alpha'_{1}, \ldots ,\alpha'_{j-1},
    \alpha'_{j+1}, \ldots ,\alpha'_{N}},
\end{align*}
but 
\[
    \zeta^{j-1}\delta_{\alpha,\alpha'_{j}}
    \ket{\alpha \alpha'_{1}, \ldots ,\alpha'_{j-1},
    \alpha'_{j+1}, \ldots ,\alpha'_{N}}
    =
    \ket{\alpha'_{1}, \ldots ,\alpha'_{N}}
\]
because $\delta_{\alpha,\alpha'_{j}}$ 
enforces 
$\alpha=\alpha'_{j}$, 
and permuting $\ket{\alpha}$ to the $j$th place 
cancels $\zeta^{j-1}$. 
There are $n_{\alpha}$ identical contributing terms, hence
\[
    \hat{\psi}^{\dagger}_{\alpha}\hat{\psi}_{\alpha}
    \ket{\alpha'_{1}, \ldots ,\alpha'_{N}}
    =
    n_{\alpha}
    \ket{\alpha'_{1}, \ldots ,\alpha'_{N}}.
\]

If $n_{\alpha}=0$ then 
\[
    \hat{\psi}_{\alpha}
    \ket{\alpha'_{1}, \ldots ,\alpha'_{N}}
    =
    \ket{\text{null}}
    =0 \cross \ket{\alpha'_{1}, \ldots ,\alpha'_{N}},
\]
and therefore the equation 
\[
    \hat{\psi}^{\dagger}_{\alpha}\hat{\psi}_{\alpha}
    \ket{\alpha'_{1}, \ldots ,\alpha'_{N}}
    =
    n_{\alpha}
    \ket{\alpha'_{1}, \ldots ,\alpha'_{N}}
\]
holds whatever $n_{\alpha}$ is. 

The operator $\hat{\psi}^{\dagger}\hat{\psi}$ is 
Hermitian because 
\[
    \left( \hat{\psi}^{\dagger}\hat{\psi} \right) ^{\dagger}
    =
    \hat{\psi}^{\dagger}
    \left( \hat{\psi}^{\dagger} \right) ^{\dagger}
    =
    \hat{\psi}^{\dagger}\psi.
\]
Comparing $\hat{\psi}^{\dagger}_{\alpha}\hat{\psi}$ with 
what was required 
in \cref{rem:from-Fock-states-to-number-operator}, 
we see that it fits the definition.

\begin{graybox}
\begin{res}
    Associated to each 1-body state $\ket{\alpha}$ 
    is a Hermitian number operator 
    \begin{equation}
        \hat{n}_{\alpha}\coloneqq
        \hat{\psi}^{\dagger}_{\alpha}\hat{\psi}_{\alpha}
    \label{eqn:definition-of-the-number-operator}
    \end{equation}
    which produces $n_{\alpha}$ when acting 
    on a Fock state: 
    \begin{equation}
        \hat{n}_{\alpha}
        \ket{n_{\lambda_1}, \ldots ,n_{\lambda_n}}
        =
        n_{\alpha}
        \ket{n_{\lambda_1}, \ldots ,n_{\lambda_n}}.
    \end{equation}
    Summing over all occupation numbers 
    gives the total particle number: 
    \begin{equation}
        \sum_{\alpha}
        \hat{n}_{\alpha}
        \ket{n_{\alpha_1}, \ldots ,n_{\alpha_{d}}}
        =
        N
    \end{equation}
    with $\{\ket{\alpha_{j}}\}_{j=1}^{d}$ 
    the 1-body basis and $N=\sum_{j=1}^{d}n_{j}$. 
\end{res}
\end{graybox}


\begin{mot}
As is customary for newly introduced operators, 
we study the commutators and how to swtich bases 
for the creation and annihilation operators. 
The formulas obtained below will be 
important tools for calculation.
\end{mot}

First, we study the commutators 
between raising operators of different 
1-body states $\ket{\alpha},\ket{\alpha'}$.
\begin{align*}
    \hat{\psi}^{\dagger}_{\alpha}
    \hat{\psi}^{\dagger}_{\alpha'}
    |\alpha_1, \ldots ,\alpha_{n}\}
    =&\ 
    \hat{\psi}^{\dagger}_{\alpha}
    |\alpha' \alpha_1, \ldots ,\alpha_{n}\} \\
    =&\ 
    |\alpha \alpha' \alpha_1, \ldots ,\alpha_{n}\} \\
    =&\ 
    \zeta 
    |\alpha' \alpha \alpha_1, \ldots ,\alpha_{n}\} \\
    =&\ 
    \zeta
    \hat{\psi}^{\dagger}_{\alpha'}
    \hat{\psi}^{\dagger}_{\alpha}
    |\alpha_1, \ldots ,\alpha_{n}\},
\end{align*}
therefore 
\[
    \comm{\hat{\psi}^{\dagger}_{\alpha}}{
        \hat{\psi}^{\dagger}_{\alpha'}
    }_{-\zeta}
    = 0,
\]
where 
\[
    \comm{\hat{A}}{\hat{B}}_{-\zeta}
    \coloneqq
    \hat{A}\hat{B}-\zeta \hat{B} \hat{A}.
\]
Similarly, 
$\comm{\hat{\psi}_{\alpha}}{\hat{\psi}_{\alpha'}}_{-\zeta}=0$.


Next we study the commutator 
between creation and annihilation operators. 
\begin{align*}
    \hat{\psi}_{\alpha}
    \hat{\psi}^{\dagger}_{\alpha'}
    |\alpha_1, \ldots ,\alpha_n\}
    =&\ 
    \hat{\psi}_{\alpha}
    |\alpha'\alpha_1, \ldots ,\alpha_{n}\} \\
    =&\ 
    \delta_{\alpha,\alpha'}
    |\alpha_1, \ldots ,\alpha_{n}\}
    +
    \sum_{j=2}^{n+1}
    \zeta^{j-1}
    \delta_{\alpha,\alpha_{j}}
    |\alpha' \alpha_1, \ldots ,
    \alpha_{j-1},\alpha_{j+1}, \ldots ,\alpha_{n}\},
\end{align*}
while applying them in the other order gives
\begin{align*}
    \hat{\psi}^{\dagger}_{\alpha'}
    \hat{\psi}_{\alpha}
    |\alpha_1, \ldots ,\alpha_{n}\}
    =&\ 
    \hat{\psi}^{\dagger}_{\alpha'}
    \sum_{j=1}^{n}
    \zeta^{j-1}
    \delta_{\alpha,\alpha_{j}}
    |\alpha_1, \ldots ,\alpha_{j-1},\alpha_{j+1},
    \cdots ,\alpha_{n}\} \\
    =&\ 
    \sum_{j=1}^{n}\zeta^{j-1}\delta_{\alpha,\alpha_{j}}
    |\alpha'\alpha_{1}\cdots \alpha_{j-1},\alpha_{j+1}, \ldots ,\alpha_{n}\}.
\end{align*}
It follows that 
\[
    \comm{\hat{\psi}_{\alpha}}{\hat{\psi}^{\dagger}_{\alpha'}}
    _{-\zeta}
    =
    \delta_{\alpha,\alpha'}.
\]


\begin{graybox}
\begin{res}
    \label{res:commutators-of-creation-and-annihilation-operators}
    For creation and annihilation operators, 
    \begin{equation}
        \comm{\hat{\psi}^{\dagger}_{\alpha}}{\hat{\psi}^{\dagger}_{\alpha'}}
        _{-\zeta}
        =
        \comm{\hat{\psi}_{\alpha}}{\hat{\psi}_{\alpha'}}
        _{-\zeta}
        =0
    \end{equation}
    \begin{equation}
        \comm{\hat{\psi}_{\alpha}}{\hat{\psi}^{\dagger}_{\alpha'}}
        _{-\zeta}
        =
        \delta_{\alpha,\alpha'}
    \end{equation}
    with 
    \begin{equation}
    \comm{\hat{A}}{\hat{B}}_{-\zeta}
    \coloneqq
    \hat{A}\hat{B}-\zeta \hat{B} \hat{A}.
    \end{equation}
    Creation and annihilation operators 
    of different modes always (anti)commute; 
    operators of the same mode 
    need more attention because their commutator 
    might produce a kronecker delta.
\end{res}
\end{graybox}

The creation and annihilation operators 
are defined 
in terms of how they act on partially normalized state. 
Such states transform as 
\[
    |\tilde{\alpha} \alpha_1 \cdots \alpha_{n}\}
    =
    \sum_{\alpha}
    \bra{\alpha}\ket{\tilde{\alpha}}
    \cross 
    |\alpha \alpha_{1}\cdots \alpha_{n}\},
\]
with $\{\ket{\tilde{\alpha}}\}$ and $\{\ket{\alpha}\}$
1-body bases. Therefore

\begin{align*}
    \hat{\psi}^{\dagger}_{\tilde{\alpha}}
    |\alpha_1\cdots \alpha_{n}\}
    =&\ 
    \sum_{\alpha}\bra{\alpha}\ket{\tilde{\alpha}}
    \cross 
    |\alpha \alpha_1, \ldots ,\alpha_{n}\} \\
    =&\ 
    \sum_{\alpha}\bra{\alpha}\ket{\tilde{\alpha}}
    \hat{\psi}^{\dagger}_{\alpha}
    |\alpha_1\cdots \alpha_{n}\},
\end{align*}
from which it follows that 
\[
    \hat{\psi}^{\dagger}_{\tilde{\alpha}}
    =
    \sum_{\alpha}\bra{\alpha}\ket{\tilde{\alpha}}
    \hat{\psi}^{\dagger}_{\alpha}.
\]
Taking the Hermitian adjoint of this equation gives 
\[
    \hat{\psi}_{\tilde{\alpha}}
    =
    \sum_{\alpha}
    \bra{\tilde{\alpha}}\ket{\alpha}
    \hat{\psi}_{\alpha}.
\]

\begin{graybox}
\begin{res}
    Given the creation and annihilation operators 
    under the 1-body basis $\{\ket{\alpha}\}$, 
    switching to another 1-body basis 
    $\{\ket{\tilde{\alpha}}\}$ leads to new 
    creation and annihilation operators 
    \begin{align*}
    \hat{\psi}^{\dagger}_{\tilde{\alpha}}
    =&\ 
    \sum_{\alpha}\bra{\alpha}\ket{\tilde{\alpha}}
    \hat{\psi}^{\dagger}_{\alpha} \\
    \hat{\psi}_{\tilde{\alpha}}
    =&\ 
    \sum_{\alpha}\bra{\alpha}\ket{\tilde{\alpha}} ^{*} \hat{\psi}_{\alpha}
    \end{align*}
    where the creation operator 
    transforms like a vector (ket), 
    and the annihilation operator 
    transforms like a 1-form (bra). 
    One may remember the formula for 
    $\hat{\psi}^{\dagger}_{\tilde{\alpha}}$ easily by noting that 
    immediately after writing down sum over $\alpha$, 
    the coefficient $\bra{\alpha}\ket{\tilde{\alpha}}$ 
    begins with $\alpha$.
\end{res}
\end{graybox}

\begin{rem}
    Switching to an infinite-dimensional 1-body basis 
    requires replacing the sum 
    with integration and 
    the Kronecker delta with 
    Dirac delta. 
    For example, 
    consider the so-called field operator
    \footnote{
        Field operators are just creation and annihilation operators 
        under the position basis.
    }
    \[
        \hat{\psi}^{\dagger}(\vb{r})
        \coloneqq
        \sum_{\alpha}\bra{\alpha}\ket{\vb{r}}
        \hat{\psi}^{\dagger}_{\alpha},
    \]
    its commutators involve Dirac (instead of Kronecker) delta functions: 
    \[
        \comm{\hat{\psi}(\vb{r})}{\hat{\psi}^{\dagger}(\vb{r'})}
        =
        \delta(\vb{r}-\vb{r'}).
    \]
    This is reasonable because 
    when evaluating expectation values, 
    these operators will be integrated over $\vb{r}$ and $\vb{r'}$, and 
    the Dirac delta can pick up a finite value while 
    the Kronecker delta cannot.
\end{rem}

\subsection{Analogy with the harmonic oscillator}

This subsection 
can be skipped. 
It is included because many texts use the 
harmonic oscillator to motivate the definition of 
the creation and annihilation operators, which 
might cause confusion.

The quantum 1-dimensional harmonic oscillator 
with the Hamiltonian 
\[
    \hat{H}
    =
    \frac{\hat{p}^{2}}{2m}
    +
    \frac{1}{2}
    m \omega ^{2} 
    \hat{x}^{2}
    =
    \hbar \omega 
    \left( \hat{a}^{\dagger} \hat{a}+\frac{1}{2} \right) 
\]
has discrete energy eigenstates 
$\left\{ \ket{n} \right\}_{n=0}^{\infty}$
related 
by the ``ladder operators'' 
\[
    \hat{a}^{\dagger}
    =
    \sqrt{\frac{m \omega}{2 \hbar}}
    \left( 
        \hat{x}
        -
        \frac{\mathrm{i}}{m \omega} 
        \hat{p}
     \right), 
    \hat{a} 
    =
    \sqrt{\frac{m \omega}{2 \hbar}}
    \left( 
        \hat{x}
        +
        \frac{\mathrm{i}}{m \omega} 
        \hat{p}
     \right)
\]
such that 
\[
    \hat{a}^{\dagger} \ket{n} = \sqrt{n+1} \ket{n+1},
    \hat{a} \ket{n} = \sqrt{n} \ket{n-1}.
\]
From this point of view, 
the quantum harmonic oscillator 
can be viewed as a many-body system 
with the 1-body Hilbert space 
$\operatorname{span}\{\ket{\text{HO}}\}
=
\mathbb{C}^{1}$. 
The ladder operators create or annihilate 
``particles'' in the 1-body state 
$\ket{\text{HO}}$. 

Recall that for Fock states, 
each 1-body state $\ket{\alpha}$ 
corresponds to 
a pair of operators 
$\hat{\psi}^{\dagger}_{\alpha}$, 
$\hat{\psi}_{\alpha}$ which 
can create or annihilate a particle in $\ket{\alpha}$. 
To form an analogy with 
a many-body system with 1-body Hilbert space 
$\mathbb{C}^{d}$, 
we need $d$ independent harmonic oscillators, 
with the wavefunction
\[
    \ket{n_1,n_2, \ldots ,n_d}
    \coloneqq
    \bigotimes _{j=1}^{d}\ket{n_{j}},
\]
where each $\ket{n_{j}}$ labels the state of one 
harmonic oscillator. 
When the $j$th harmonic oscillator 
is excited by $\hat{a}^{\dagger}_{j}$, 
\textit{i.e.}\ when the wavefunction 
$\ket{n_1,n_2, \ldots ,n_d}$ 
is acted upon 
by this operator, 
we \emph{think about this event as if} 
a boson is created in the 1-body state 
corresponding to the $j$th harmonic oscillator. 
This is the analogy with the harmonic oscillator; 
it arrives at the bosonic Fock state 
via a specific hamiltonian. 

Had we introduced second quantization 
with this analogy, we would not be able to know 
whether the formalism depends on 
the harmonic oscillator in some fundamental way, 
or whether the second quantization for fermions 
can also be constructed with an analogy. 
Besides, the state $\ket{\text{HO}}$ is 
totally abstract; we do not know what it is. 

However, our discussions prior to this subsection 
were \emph{altogether independent of any Hamiltonian}. 
It is clear from our approach that second quantization 
does not depend on the harmonic oscillator.

But this analogy does reveal that 
the harmonic oscillator is something special. 
In fact, the quantized electromagnetic field 
turns out to be a sum of harmonic oscillators 
whose excitations are photons.

\section{Operators under second quantization}

\begin{mot}
    The key difficulty is the conflict between our basis 
    and the way we make sense of operators. We are only familiar 
    with the way in which operators act on product states, but 
    we are now using Fock states.    
    To resolve this 
    we project Fock states onto the product basis 
    and act the operators on them. After that we 
    collect the terms. The general recipe would be 
    to act an operator on a 
    ``partially normalized state'' 
    and then abstract away the state 
    \footnote{
        While not normalized, the basis 
        $\{ |\alpha_1, \ldots ,\alpha_{N}\} \}$
        is still an orthogonal and complete one.
    }
    to obtain an operator equation.
\label{mot:operators-under-second-quantization}
\end{mot}

\subsection{1-body operators}

By definition, a 1-body operator $\hat{V}$ visits 
the particles in a product state one by one. 
Under a basis where $\hat{V}$ is diagonal, \textit{i.e.}~
within each 1-body space, 
$\hat{V}
=
\sum_{j=1}^{d}\lambda_{j} 
\ket{\lambda_{j}}
\bra{\lambda_{j}}
$, acting it on a product state would produce 

\begin{align*}
    \hat{V}
    |\alpha_1, \ldots ,\alpha_{N})
    =&\ 
    \sum_{i = 1}^{N}
    \hat{V}^{(i)}
    |\alpha_1, \ldots ,\alpha_{N}) \\
    =&\ 
    \sum_{i = 1}^{N}
    \alpha_{i}
    |\alpha_1, \ldots ,\alpha_{N})
\end{align*}
where $\alpha_{i}\in \{\lambda_{j}\}$ and 
the superscript $(i)$ on $\hat{V}$ 
indexes the 1-body space $\hat{V}$ acts on. 
Note that the sum $\sum_{i=1}^{N}\alpha_{i}$ 
only depends on occupation numbers, hence 
it is invariant upon permutation of particles.

Following the recipe mentioned 
in \cref{mot:operators-under-second-quantization}
we calculate $\hat{V}|\alpha_1, \ldots ,\alpha_{N}\}$,

\begin{align*}
    \hat{V}|\alpha_1, \ldots ,\alpha_{N}\}
    =&\ 
    \frac{1}{\sqrt{N!}}
    \sum_{P}\zeta^{\abs{P}}
    \hat{V} |\alpha_{P1}, \ldots ,\alpha_{PN}\} \\
    =&\ 
    \frac{1}{\sqrt{N!}}
    \sum_{P}\zeta^{\abs{P}}
    \sum_{i =1} ^{N}
    \hat{V}^{(Pi)}
    |\alpha_{P1}, \ldots ,\alpha_{PN}\} \\
    =&\ 
    \frac{1}{\sqrt{N!}}
    \sum_{P}\zeta^{\abs{P}}
    \sum_{i = 1}^{N}
    \alpha_{Pi} |\alpha_{P1}, \ldots ,\alpha_{PN}\} \\
    =&\ 
    \frac{1}{\sqrt{N!}}
    \sum_{P}\zeta^{\abs{P}}
    \left( 
        \sum_{j=1}^{d}
        \hat{n}_{\lambda_{j}}
        \lambda_{j}
    \right) 
    |\alpha_{P1}, \ldots ,\alpha_{PN}\} \\
    =&\ 
    \sum_{j=1}^{d}
    \hat{n}_{\lambda_{j}}
    \lambda_{j}
    |\alpha_1, \ldots ,\alpha_{N}\},
\end{align*}
where the last equality holds 
because in the second-to-last line, 
the term $\sum_{j}\hat{n}_{j}\lambda_{j}$ 
is independent of $P$ and thus can be factorized. 

Switching basis should follow from straightforward 
calculation: start with 
\[
    \hat{V}^{(1)}
    =
    \sum_{j=1}^{d}
    \bra{j} \hat{V} \ket{j}
    \hat{\psi}^{\dagger}_{j}\hat{\psi}_{j},
\]
where we have denoted 
$\ket{\lambda_{j}}$ with $\ket{j}$, 
and insert 
\[
    \hat{\psi}^{\dagger}_{j}
    =
    \sum_{i'} \bra{i'} \ket{j}
    \hat{\psi}^{\dagger}_{i'}
    ,\quad
    \hat{\psi}_{j}
    =
    \sum_{k'} \bra{k'} \ket{j} ^{*}
    \hat{\psi}_{k'}
\]
to obtain
\begin{align*}
    \hat{V}^{(1)}=&\ 
    \sum_{j,i',k'}
    \bra{j}\hat{V} \ket{j}
    \hat{\psi}^{\dagger}_{i'}\hat{\psi}_{k'}
    \bra{i'} \ket{j} \bra{j} \ket{k'}
    \\
    =&\ 
    \sum_{j,i',k'}
    \bra{i'}\ket{j}\bra{j} \hat{V} \ket{j} \bra{j} \ket{k'}
    \hat{\psi}^{\dagger}_{i'}
    \hat{\psi}_{k'}
    \\
    =&\ 
    \sum_{i',k'}
    \bra{i'} \hat{V} \ket{k'}
    \hat{\psi}^{\dagger}_{i'} \hat{\psi}_{k'},
\end{align*}
where the last equation holds because 
summing over $j$ in $\ket{j} \bra{j}$ produces $\mathds{1}$.

\begin{graybox}
\begin{res}
    \label{res:1-body-operators}
    Under second quantization, 
    a 1-body operator $\hat{V}$ 
    takes the form of
    \begin{equation}
        \hat{V}
        =
        \sum_{i,j=1}^{d}
        \hat{\psi}_{\lambda_{i}}^{\dagger}
        \hat{\psi}_{\lambda_{j}}
        \bra{\lambda_{i}} \hat{V} 
        \ket{\lambda_{j}},
    \label{eqn:1-body-operator-under-second-quantization}
    \end{equation}
    where $\{\ket{\lambda_{j}}\}_{j=1}^{d}$ 
    is any basis
    for the 1-body Hilbert space $\mathbb{C}^{d}$. 
    If $\left\{ \ket{\lambda_{j}} \right\}$ 
    is an eigenbasis of $\hat{V}$ then 
    we can insert $i = j$ and sum over $i$.
\end{res}
\end{graybox}

\begin{rem}
    \label{rem:recipe-for-calculating-2nd-quantized-operators}
    From the calculations that led to 
    \cref{res:1-body-operators}, 
    we can see the general recipe of 
    deriving operators under second quantization: 
    \begin{enumerate}
        \item 
        Because $\hat{V}$ acting on 
        a product state produces 
        a scalar
        $\sum_{i}\alpha_{i}$
        that is 
        independent of permutation, 
        this scalar can be factorized 
        out of the sum over permutations.
        \item
        Because 
        $\forall$ 
        $f: \mathbb{C}^{d} \mapsto \mathbb{C}$ 
        and a Fock state 
        $\ket{\alpha_1, \ldots ,\alpha_{N}}$ 
        with 1-body Hilbert space $\mathbb{C}^{d}$
        we have 
        \[
            \sum_{i = 1}^{N}
        f(\ket{\alpha_{i}})
        =
        \sum_{j=1}^{d}
        n_{\lambda_{j}}f(\ket{\lambda_{j}}),
        \]
        we can express $\hat{V}$ 
        in terms of the number operator.
    \end{enumerate}
\end{rem}

\begin{rem}
    The only information carried by a 
    physical state is occupation numbers, 
    so it is only natural that 
    any Hermitian operator can be 
    expressed in terms of 
    number operators. 
\end{rem}

\subsection{2-body operators}

By definition, 2-body operators act on 
\emph{pairs} of particles only. 
For a 2-body operator $\hat{U}$, under its 
eigenbasis 
$\{\ket{\lambda_{j}}\}_{j=1}^{d}$, 
$\hat{U}=\sum_{p < q}\hat{U}^{(p,q)}$ where  
\[
    \hat{U}^{(p,q)}=\hat{U}^{(p)}\otimes \hat{U}^{(q)}
\]
with 1-body operators 
$\hat{U}^{(p)}=\hat{U}^{(q)}=
\sum_{j=1}^{d}\lambda_{j} \ket{\lambda_{j}}
\bra{\lambda_{j}}$. 

Therefore, acting $\hat{U}$ on a product state gives 
\begin{align*}
    \hat{U}
    |\alpha_1, \ldots ,\alpha_{N})
    =&\ 
    \sum_{i<j}
    \hat{U}^{(i,j)}
    |\alpha_1, \ldots ,\alpha_{N}) \\
    =&\ 
    \sum_{i<j}
    \alpha_{i}\alpha_{j}
    |\alpha_1, \ldots ,\alpha_{N})
\end{align*}
with 
$\alpha_{i},\alpha_{j} 
\in 
\{\lambda_{k}\}_{k=1}^{d}$.

We apply the recipe given in 
\cref{rem:recipe-for-calculating-2nd-quantized-operators},

\begin{enumerate}
    \item Factorizing the scalar,
    \begin{align*}
        \hat{U} 
        |\alpha_1, \ldots ,\alpha_{N}\}
        =&\ 
        \frac{1}{\sqrt{N!}}
        \sum_{P}\zeta^{\abs{P}}
        \hat{U}
        |\alpha_{P1}, \ldots ,\alpha_{PN}) \\
        =&\ 
        \left( 
            \sum_{i<j}
            \alpha_{i}\alpha_{j}
         \right) 
        |\alpha_{1}, \ldots ,\alpha_{N}\},
    \end{align*}
    \item Expressing the scalar 
    in terms of occupation numbers,
    \begin{align*}
        \sum_{i<j}
        \alpha_{i}\alpha_{j}
        =&\ 
        \frac{1}{2}
        \left( 
            \sum_{i,j=1}^{N}
            -
            \sum_{i=j}
         \right) 
        \alpha_{i}\alpha_{j} \\
        =&\ 
        \frac{1}{2}
        (\sum_{i}\alpha_{i})
        (\sum_{j}\alpha_{j})
        -
        \frac{1}{2}\sum_{i}\alpha_{i}^{2} \\
        =&\ 
        \frac{1}{2}\sum_{\lambda,\lambda'}
        \lambda \lambda' 
        n_{\lambda} n_{\lambda '}
        -
        \frac{1}{2}
        \sum_{\lambda} n_{\lambda} \lambda^{2} \\
        =&\ 
        \frac{1}{2}\sum_{\lambda,\lambda'}
        \lambda \lambda '
        (
            n_{\lambda} n_{\lambda '}
            -
            \delta_{\lambda,\lambda '}
            n_{\lambda}
        ),
    \end{align*}
    where $\lambda, \lambda'$ label 
    1-body eigenvalues,
\end{enumerate}

to obtain the 
second-quantized formula for 2-body operators, 
\[
    \hat{U}
    =
    \frac{1}{2}
    \sum_{\lambda,\lambda'}
    \lambda \lambda'
    \left( \hat{n}_{\lambda}\hat{n}_{\lambda'}
    -
    \delta_{\lambda,\lambda'}
    \hat{n}_{\lambda} \right) .
\]

In fact, $\hat{n}_{\alpha}\hat{n}_{\beta}
    -
    \delta_{\alpha,\beta}
    \hat{n}_{\alpha}$ 
    equals to $\hat{\psi}^{\dagger}_{\alpha}\hat{\psi}^{\dagger}_{\beta}
    \hat{\psi}_{\beta}\hat{\psi}_{\alpha}$. 
If $\alpha \neq \beta$ then this equation is trivial; 
otherwise, we can verify this using 
\cref{res:commutators-of-creation-and-annihilation-operators}: 
for Bosons, 
\[
    \psi ^{\dagger} \psi \psi ^{\dagger} \psi
    -
    \psi ^{\dagger} \psi
    =
    \psi ^{\dagger} (\psi \psi ^{\dagger} -\mathds{1}) \psi
    = \psi ^{\dagger} \psi ^{\dagger} 
    \psi \psi,
\]
and for fermions $\psi ^{\dagger} \psi ^{\dagger} \psi \psi=0$, while
\[
    \psi ^{\dagger} \psi \psi ^{\dagger} \psi 
    -
    \psi ^{\dagger} \psi
    =
    \psi ^{\dagger} (\psi \psi ^{\dagger} -\mathds{1}) \psi
    =
    - \psi ^{\dagger} \psi ^{\dagger} \psi \psi =0.
\]

Therefore, under second quantization, 
a 2-body operator $\hat{U}$ is 
\begin{equation*}
    \hat{U}
    =
    \frac{1}{2}
    \sum_{\lambda,\lambda'}
    (\lambda \lambda'
    |\hat{U}|
    \lambda \lambda')
    \hat{\psi}^{\dagger}_{\lambda}
    \hat{\psi}^{\dagger}_{\lambda'}
    \hat{\psi}_{\lambda'}
    \hat{\psi}_{\lambda}.
\end{equation*}

Concerning switching basis, 
if we take the straightforward approach 
of inserting the 
transformation for $\hat{\psi}^{\dagger}_{\lambda}$, 
$\hat{\psi}^{\dagger}_{\lambda'}$, 
$\hat{\psi}_{\lambda'}, \hat{\psi}_{\lambda}$, then 
the calculation would be tedious. 
Instead we reduce the problem at hand 
to one already solved: switching basis 
for 1-body operators. 
In \cref{eqn:1-body-operator-under-second-quantization} 
the basis is \textit{arbitrary}, meaning that 
this equation is formally basis independent. 
If we can write down 2-body operators 
in terms of that equation, then 
their expression become formally basis independent, too.

We now 
re-write $\hat{U}$ 
in terms of 1-body operators. 
Recalling $\hat{U}$ is a sum of 
terms like $\hat{U}^{(p)}\otimes \hat{U}^{(q)}$, 
the inner product 
$
    (\lambda \lambda'
    |\hat{U}|
    \lambda \lambda')
$
is just a sum of terms like
\begin{align*}
    (\lambda \lambda' | \hat{U}^{(p)}\otimes \hat{U}^{(q)}
    |\lambda \lambda')
    =&\ 
    (\lambda \lambda' | \hat{U}^{(p)}\otimes \mathds{1}^{(q)}
    |\lambda \lambda')
    \cross 
    (\lambda \lambda' | \mathds{1}^{(p)}\otimes \hat{U}^{(q)}
    |\lambda \lambda')
    \\
    =&\ 
    \bra{\lambda} \hat{U}^{(p)} \ket{\lambda}
    \cross 
    \bra{\lambda'} \hat{U}^{(q)} \ket{\lambda'},
\end{align*}
which are 1-body quantities. 

Having taken care of the scalar part
$
    (\lambda \lambda'
    |\hat{U}|
    \lambda \lambda'),
$ 
let us examine the operator part of $\hat{U}$. In the equation
\[
    \hat{U}=
    \frac{1}{2}
    \sum_{\lambda,\lambda'}
    \lambda \lambda'
    \left( \hat{n}_{\lambda}\hat{n}_{\lambda'}
    -
    \delta_{\lambda,\lambda'}
    \hat{n}_{\lambda} \right) .
\]
there are two terms: the first one is 
\begin{align*}
    (\lambda \lambda'
    |\hat{U}|
    \lambda \lambda')
    \hat{n}_{\lambda} \hat{n}_{\lambda'}
    =&\ 
    \bra{\lambda} \hat{U}^{(p)} \ket{\lambda}
    \hat{n}_{\lambda}
    \cross 
    \bra{\lambda'} \hat{U}^{(q)} \ket{\lambda'}
    \hat{n}_{\lambda'},
\end{align*}
and the second is zero for $\lambda \neq \lambda'$, 
\[
    \bra{\lambda} \hat{U}^{(p)} \ket{\lambda}
    \bra{\lambda} \hat{U}^{(q)} \ket{\lambda}
    \hat{n}_{\lambda}
    =
    \expval{\left( \hat{U}^{(1)} \right)^{2}}{\lambda}
    \hat{n}_{\lambda}
\]
for $\lambda = \lambda'$ where 
$\hat{U}^{(p)}=\hat{U}^{(q)}=\hat{U}^{(1)}$. 
Both terms consist of 
1-body number operators multiplied by 
their corresponding eigenvalues, which 
match the form of 
\cref{eqn:1-body-operator-under-second-quantization}, 
so they are formally basis independent. 

\begin{graybox}
    \begin{res}
        Under second quantization, 
        a 2-body operator $\hat{U}$ is 
        \begin{equation}
            \hat{U}
            =
            \frac{1}{2}
            \sum_{\alpha,\beta,\alpha',\beta'}^{d}
            (\alpha \beta
            |\hat{U}|
            \alpha' \beta')
            \hat{\psi}^{\dagger}_{\alpha}
            \hat{\psi}^{\dagger}_{\beta}
            \hat{\psi}_{\beta'}
            \hat{\psi}_{\alpha'},
        \end{equation}
        with 
        $\ket{\alpha},\ket{\beta}, 
        \ket{\alpha'}, \ket{\beta'}
        \in \{\ket{\lambda}\}$, 
        an arbitrary basis 
        for the 1-body space $\mathbb{C}^{d}$. 
        If 
        $\{|\alpha \beta)\}$ 
        is an eigenbasis of $\hat{U}$, then 
        we can insert 
        $\alpha = \alpha', \beta = \beta'$ 
        and sum over $\alpha, \beta$.
    \end{res}
\end{graybox}

\section{General comments and bibliographical notes}
While important for any advanced study in physics, 
second quantization has not been given the rigor and 
emphasis it deserves in many texts. Most textbooks simply 
use the harmonic oscillator to motivate the formulae of 
creation and annihilation operators, and force the reader 
to accept Fock states without explicitly constructing them 
using product states.

The approach taken in this note is better than most books 
since it shows that second quantization is simply a basis 
transformation within physical subspaces from product states to 
Fock states (which are ``scrambled'' product states). 
There is hardly any physical subtlety or postulate 
related with second quantization, except perhaps 
that 
the physical nature of ``permutation'' depends on 
dimensionality, which gives rise to anyons in (2+1)D. 
\footnote{
    A simple argument concerning 
    how particle exchange statistics 
    are fundamentally different between 
    (2+1)D and (3+1)D can be found in 
    Chapter 3 of \cite{simon_topological_2023}.
    Click on the hyperlink provided 
    in the bibliography to access 
    the free version of the book.
}

This note treats 
the topic so that 
\blockcquote{shankar_principles_1994}{by segregating mathematical theorems from physical postulates, any confusion as to which is which is nipped in the bud}. 
While not particularly enthusiastic about formal topics in physics, 
I find this slightly formal introduction to second quantization suitable.


\printbibliography
\end{document}
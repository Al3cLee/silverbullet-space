\documentclass{article}

% document structure
\usepackage[hidelinks]{hyperref}
\usepackage[backend=biber]{biblatex}
\addbibresource{ref.bib}
\usepackage[margin=1in]{geometry}

% advanced lists
\usepackage{enumitem}
\newlist{steps}{enumerate}{2}
\setlist[steps]{label=\textit{Step \arabic*}., 
leftmargin=*, ref = Step \arabic*}

% math formatting
\usepackage{xcolor,dsfont,amsthm,amsfonts,mathtools,amsmath,physics}
\theoremstyle{definition}
\newtheorem{mot}{Motivation}[section]
\newtheorem{rem}{Remark}[section]
\newtheorem{dfn}{Definition}[section]
\newtheorem{eg}{Example}[section]
\newtheorem{recipe}{Recipe}[section]
\theoremstyle{plain}
\newtheorem{thm}{Theorem}[section]
\newtheorem{lem}{Lemma}[section]
\newtheorem{res}{Result}[section]

\numberwithin{equation}{section}

\usepackage{cleveref}
\crefname{figure}{fig.}{figs.}
\crefname{equation}{eqn.}{eqns.}
\crefname{dfn}{definition}{definitions}
\crefname{mot}{motivation}{motivations}
\crefname{thm}{theorem}{theorems}
\crefname{lem}{lemma}{lemmas}
\crefname{rem}{remark}{remarks}
\crefname{res}{result}{results}
\crefname{sec}{section}{sections}

\author{Wentao Li}
\begin{document}
    
\title{Projectors}

\maketitle

\begin{abstract}
    This note 
    loosely follows 
    \cite{nielsen_quantum_2010} and 
    explains basic properties of projectors 
    in the context of linear algebra. 
    Prerequisites include 
    subspaces and unitary operators. 
    The last section assumes an understanding 
    of the eigenvalue problem of Hermitian operators.

    A trick of constructing a projector onto a subspace 
    without explicitly knowing its basis is given in \cref{sec:a-useful-trick}. 
    This trick is 
    commonly used in physics.
\end{abstract}

\tableofcontents

\section{Definition and switching basis}

\begin{mot}
    We will define projectors 
    \textit{in a basis-independent manner} 
    and see that they are useful tools 
    for dealing with subspaces. 
    In fact a subspace and the projector onto it 
    carry virtually the same amount of information, 
    and for this reason 
    we often use one of them to denote the other. 
\end{mot}

\begin{dfn}
    For a $d$-dimensional vector space $\mathbb{V}$, 
    the \textit{projector} 
    $P_{\mathbb{W}}$ onto a subspace 
    $\mathbb{W}$ of $\mathbb{V}$ is 
    an operator such that for any vector 
    $\ket{v}\in \mathbb{V}$, 

    \begin{itemize}
        \item 
        If $\ket{v}\in \mathbb{W}$ then 
        $P_{\mathbb{W}} \ket{v}=\ket{v}$,
        \item 
        If $\ket{v} \bot \mathbb{W}$ then 
        $P_{\mathbb{W}} \ket{v}=0$.
    \end{itemize}
    \label{dfn:projector}
\end{dfn}

\begin{lem}
\label{lem:equivalence-between-projector-and-subspace}
    A vector $\ket{w}$ belongs to $\mathbb{W}$ if and only if 
    $P_{\mathbb{W}} \ket{w} = \ket{w}$.
\end{lem}

\begin{proof}
    By definition, $\ket{w} \in \mathbb{W}$ implies 
    $P_{\mathbb{W}}=\ket{w}$. We need only establish implication 
    in the other direction. We prove by its contrapositive. 

    If $\ket{v} \notin \mathbb{V}$ then we write 
    $\ket{v} = \ket{v'} + \ket{\theta}$ with 
    $\ket{v'} \in \mathbb{V}$, 
    $\ket{\theta} \bot \mathbb{V}$ and $\ket{\theta} \neq \ket{\text{null}}$. 
    It follows that $P_{\mathbb{W}} \ket{v} = \ket{v'} \neq \ket{v}$, hence 
    the contrapositive is proven.
\end{proof}

\begin{rem}
    From \cref{lem:equivalence-between-projector-and-subspace} we see that 
    a subspace and the projector onto it 
    carry the same amount of information. They both specify the subspace.
\end{rem}

\begin{lem}
    Given an orthonormal basis 
    $\left\{ \ket{w_{i}} \right\}_{i =1}^{k}$ 
    of $\mathbb{W}$, the projector onto $\mathbb{W}$ 
    can be constructed as 
    \begin{equation}
        P_{\mathbb{W}} 
        = 
        \sum_{i = 1}^{k} \ket{w_i} \bra{w_i}.
    \label{eqn:projector-construction-from-basis}
    \end{equation}
\end{lem}

\begin{proof}
    We only need to check 
    whether this explicit constrction fits 
    \cref{dfn:projector}.
    For $\ket{v}\in \mathbb{W}$, write 
    $\ket{v}= \sum_{j=1}^{k}v_j \ket{w_j}$. Then 
    \begin{equation}
        P_{\mathbb{W}}\ket{v}
        =
        \sum_{j=1}^{k}\sum_{i = 1}^{k}
        \ket{w_i}\bra{w_i} v_j \ket{w_j} 
        =
        \sum_{i,j} \delta_{i,j} v_j \ket{w_i} = \ket{v}.
    \end{equation}

    For $\ket{v'} \bot \mathbb{W}$, write 
    $\ket{v'} = \sum_{j=1}^{d-k} v'_j \ket{u_j}$ 
    with $\ket{u_j} \bot \mathbb{W}$, 
    \begin{equation}
        P_{\mathbb{W}}\ket{v'}
        =
        \sum_{i,j} \ket{w_i} \bra{w_i} v'_j \ket{u_j}
        =
        0.
    \end{equation}
    Now the lemma follows.
\end{proof}

\begin{lem}
    Using \cref{eqn:projector-construction-from-basis} 
    to define projectors is also acceptable, 
    because the equation is invariant upon 
    change of basis. 
\end{lem}

\begin{rem}
    This is natural, 
    since our definition of projectors only involves 
    the subspace. 
    There is no reason to expect the projector 
    to depend on any basis. 
    On the other hand, had we defined the projector 
    with \cref{eqn:projector-construction-from-basis}, 
    we would need to 
    \textit{explicitly verify} its invariance 
    under change of basis.
\end{rem}

\begin{proof}
    First, we need to re-phrase this claim. 
    For a projector 
    $P = \sum_{i = 1}^{k} \ket{i} \bra{i}$
    onto a subspace, supposed we choose another basis 
    $\left\{ \ket{\tilde{j}} \right\}_{j = 1}^{k}$ 
    of this subspace to construct the projector 
    $P' = 
    \sum_{j = 1}^{k} \ket{\tilde{j}} \bra{\tilde{j}}$.

    Now, in order to prove that 
    $P' = P$, 
    we take an arbitrary vector 
    $\ket{\phi} = 
    \sum_{m =1}^{d}c_m \ket{m}$, 
    then expand $P' \ket{\phi}$ and 
    $P \ket{\phi}$ under the same basis. 
    Here we take 
    $\left\{ \ket{m} \right\}_{m = 1}^{d}$ 
    so that this basis coincides with 
    $\left\{ \ket{i} \right\}_{i = 1}^{k}$ 
    within the subspace 
    $\operatorname{span} \{ \ket{1}, \ldots , \ket{k}\}$. 

    Expanding as mentioned we have 
    \begin{equation}
        P \ket{\phi}
        =
        \sum_{i = 1}^{k}
        \sum_{m = 1}^{k}
        c_m \ket{i} \bra{i} \ket{m}
        = \sum_{i = 1}^{k} c_i \ket{i},
    \end{equation}
    \begin{align}
        P' \ket{\phi} 
        =&\ 
        \sum_{j=1}^{k}\sum_{m=1}^{d}
        c_m \ket{\tilde{j}} \bra{\tilde{j}} \ket{m} \\
        =&\ 
        \sum_{j=1}^{k}
        \left( 
            \sum_{m=1}^{k}
            +
        \mathcolor{blue}{
            \sum_{m=k+1}^{d}
        }
         \right) 
        c_m \ket{\tilde{j}} \bra{\tilde{j}} \ket{m} \\
        =&\ 
        \sum_{j=1}^{k}\sum_{m=1}^{k}
        c_m \ket{\tilde{j}} \bra{\tilde{j}} \ket{m},
    \label{eqn:expansion-of-P-prime-phi-ket}
    \end{align}
    where the blue term does not contribute because 
    $\forall k+1 \le m \le d,\quad 
    \bra{\tilde{j}} \ket{m} =0$.
    
    Suppose the two bases are related by a unitary 
    $T$ such that 
    $\ket{\tilde{j}} = 
    \sum_{\ell=1}^{k}T_{j \ell} \ket{\ell}$. 
    Then 
    $\bra{\tilde{j}}
    =
    \sum_{p}T_{jp}^{*} \bra{p}$, and 
    inserting this into eqn.\ 
    \ref{eqn:expansion-of-P-prime-phi-ket} gives 
    \begin{align}
        P' \ket{\phi} =&\ 
        \mathcolor{blue}{\sum_{j=1}^{k}}
        \sum_{\ell, p=1}^{k}
        \mathcolor{blue}{T_{j \ell}T_{jp}^{*}}
        \ket{\ell} \bra{p}
        \sum_{m=1}^{k} c_m \ket{m} \\
        =&\ 
        \sum_{\ell,p=1}^{k}
        \mathcolor{blue}{\delta_{\ell,p}}
        \sum_{m=1}^{k} c_m \ket{m} 
        = \sum_{\ell=1}^{k} c_{\ell} \ket{\ell}. 
    \end{align}
    This completes our expansion 
    and finishes the proof.
\end{proof}
\section{Idempotence and orthogonal complement}
\begin{mot}
    We now gain more intuition into 
    projector by showing some of their properties. 
    First, projectors are idempotent, 
    \textit{i.e.}~applying them more than once 
    is the same as applying them only once. 
    This is natural because projectors make a binary 
    decision concerning the vector they act on, 
    and they only need to make this decision once. 
    Secondly, given a projector onto subspace 
    $\mathbb{V}$, we can quickly obtain the 
    projector onto its orthogonal complement 
    $\{\ket{w}: \ket{w} \bot \mathbb{V}\}$, 
    because a projector is just 
    \emph{the identity within some subspace}.
    \label{mot:idempotence-and-orthogonal-complement}
\end{mot}

\begin{thm}
    If $P$ is a projector then $P^{2}=P$.
\end{thm}
\begin{proof}
    We work under a basis 
    $\{\ket{i}\}_{i = 1}^{k}$. 
    Expanding $P^{2}$ gives 
    \begin{equation}
        P^{2}
        =
        \sum_{j,\ell = 1}^{k}
        \ket{j}\bra{j}\ket{\ell}\bra{\ell}
        =
        \sum_{j,\ell=1}^{k}
        \delta_{j,\ell}
        \ket{j}\bra{\ell}
        =P,
    \end{equation}
    and our lemma follows.
\end{proof}

\begin{rem}
    This proof exploits the orthonormality of 
    basis vectors within the subspace $P$ 
    projects onto, but fails to capture the 
    intuition mentioned in 
    \cref{mot:idempotence-and-orthogonal-complement}. 
\end{rem}

\begin{mot}
    We defined projectors to be operators which 
    preserve operators in a subspace and kills 
    vectors orthogonal to this subspace. The 
    opposite effect is killing vectors within 
    this subspace and preserving vectors orthogonal 
    to it. This ``opposite effect'' also sounds like 
    a projector. Let us construct it. 
\end{mot}

\begin{dfn}
    Given a projector $P$ onto a subspace $\mathbb{V}$, 
    its \emph{orthogonal complement} $Q$ is 
    the projector onto the subspace 
    $\{\ket{q}: \ket{q} \bot \mathbb{V}\}$. 
\end{dfn}

\begin{thm}
    The orthogonal complement $Q$ of a projector 
    $P$ is 
    \begin{equation}
        Q=\mathds{1} - P.
    \end{equation}
\end{thm}

\begin{proof}
    We can check this explicitly. 
    \begin{itemize}
        \item 
        If $\ket{v} \in \mathbb{V}$ then 
        $Q \ket{v} = \ket{v}-P \ket{v} = 0$, 
        \item 
        If $\ket{w} \bot \mathbb{V}$ then 
        $Q \ket{w} = \ket{w} - P \ket{w} 
        = \ket{w} - 0 = \ket{w}$,
    \end{itemize}
    so $\mathds{1} - P$ fits the definition of 
    orthogonal complement.
\end{proof}

\section{A useful trick}
\label{sec:a-useful-trick}

\begin{mot}
    Our definition of projectors was basis independent, 
    so in principle there should exist other methods to 
    construct them than that given by 
    \cref{eqn:projector-construction-from-basis}. 
    Here we give an example of such a method.
\end{mot}

\begin{lem}
    Given a Hermitian operator $M$ with 
    eigenvalues $\{+1,-1\}$, 
    the projector $P_{+1}$ onto the 
    invariant subspace $\mathbb{V}_{+1}$ is 
    \begin{equation}
        P_{+1}=\frac{1}{2}(\mathds{1} + M). 
    \label{eqn:projector-onto-plus-one-subspace}
    \end{equation}
    The validity of this projector can be checked easily. 
    Note that this example works because $M$ is Hermitian; 
    had $M$ been an arbitrary normal matrix (\textit{e.g.}~a unitary), 
    then this method will not work.
\end{lem}

\begin{rem}
    This trick is very useful in physics, because 
    the Pauli matrices are Hermitians with 
    eigenvalues $\{ \pm 1\}$. Various observables are built 
    from these matrices, and they therefore satisfy the 
    condition for using this trick. 
\end{rem}

\printbibliography

\end{document}

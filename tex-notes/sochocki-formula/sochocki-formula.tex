\documentclass{article}

% document structure
\usepackage[hidelinks]{hyperref}
\usepackage{csquotes}
% \blockcquote[⟨prenote⟩][⟨postnote⟩]{⟨key⟩}[⟨punct⟩]{⟨text⟩}⟨tpunct⟩
\usepackage[backend=biber]{biblatex}
\addbibresource{ref.bib}
\usepackage[margin=1in]{geometry}

% advanced lists
\usepackage{enumitem}
\newlist{steps}{enumerate}{2}
\setlist[steps]{label=\textit{Step \arabic*}., 
leftmargin=*, ref = Step \arabic*}

% fancy boxes
\usepackage{tcolorbox}
\newtcolorbox{graybox}{colback=gray!20!white, boxrule=0mm, sharp corners=all}

% math formatting
\usepackage{xcolor,dsfont,amsthm,amsfonts,mathtools,amsmath,physics}
\usepackage[mathcal]{euscript}
\theoremstyle{definition}
\newtheorem{pos}{Postulate}[section]
\newtheorem{res}{Result}[section]
\newtheorem{mot}{Motivation}[section]
\newtheorem{calc}{Calculation}[section]
\newtheorem{rem}{Remark}[section]
\newtheorem{dfn}{Definition}[section]
\newtheorem{eg}{Example}[section]
\newtheorem{recipe}{Recipe}[section]
\theoremstyle{plain}
\newtheorem{thm}{Theorem}[section]
\newtheorem{lem}{Lemma}[section]
\numberwithin{equation}{section}

\usepackage{cleveref}
\crefname{figure}{fig.}{figs.}
\crefname{equation}{eqn.}{eqns.}
\crefname{dfn}{definition}{definitions}
\crefname{mot}{motivation}{motivations}
\crefname{thm}{theorem}{theorems}
\crefname{lem}{lemma}{lemmas}
\crefname{rem}{remark}{remarks}
\crefname{calc}{calculation}{calculations}
\crefname{res}{result}{results}
\crefname{sec}{section}{sections}

\author{Wentao Li}
\begin{document}
    
\title{The Sochocki Formula}

\maketitle

\begin{abstract}
    The Sochocki formula (proven by 
    Polish mathematician Julian Karol Sochocki) 
    is stated and proven. 
    This formula is also known as the 
    ``Dirac relation'' in physics literature.
\end{abstract}

We want to derive the distribution identity
\begin{equation}
\frac{1}{x-i0^+} = \mathcal{P}\frac{1}{x} + i\pi \delta(x).
\end{equation}

This should be understood in the sense of distributions, i.e.\ after integrating against a smooth test function $f(x)$.

\section{Real part: principal value}

Consider
\begin{equation}
\Re \frac{1}{x-i\eta} = \frac{x}{x^{2}+\eta^{2}}, \qquad \eta > 0.
\end{equation}
We want to show
\begin{equation}
\lim_{\eta\to0^+}\int_{-\infty}^{\infty} f(x)\,\frac{x}{x^2+\eta^2}\,dx
= \mathcal{P}\int_{-\infty}^{\infty}\frac{f(x)}{x}\,dx.
\end{equation}

Write
\begin{equation}
f(x)=f(0)+x g(x), \qquad g(x)=\frac{f(x)-f(0)}{x}, \quad (x\neq 0).
\end{equation}
Then
\begin{align}
\int f(x)\frac{x}{x^2+\eta^2}\,dx
&= f(0)\int \frac{x}{x^2+\eta^2}\,dx + \int g(x)\frac{x^2}{x^2+\eta^2}\,dx, \\
&= 0 + \int g(x)\frac{x^2}{x^2+\eta^2}\,dx.
\end{align}
As $\eta \to 0$, $\frac{x^2}{x^2+\eta^2} \to 1$, so by dominated convergence
\begin{equation}
\lim_{\eta\to0^+}\int g(x)\frac{x^2}{x^2+\eta^2}\,dx = \int g(x)\,dx.
\end{equation}
Meanwhile,
\begin{equation}
\mathcal{P}\int \frac{f(x)}{x}\,dx = \int g(x)\,dx,
\end{equation}
since the constant term vanishes in the principal value sense. Thus the real part gives the PV integral.

\section{Imaginary part: delta function}

Now consider
\begin{equation}
\Im \frac{1}{x-i\eta} = \frac{\eta}{x^2+\eta^2}.
\end{equation}
We compute
\begin{equation}
\int_{-\infty}^{\infty} f(x)\,\frac{\eta}{x^2+\eta^2}\,dx.
\end{equation}
Change variables $x = \eta t$:
\begin{equation}
\int_{-\infty}^{\infty} \frac{f(\eta t)}{1+t^2}\,dt.
\end{equation}
As $\eta \to 0$, $f(\eta t)\to f(0)$, so
\begin{equation}
\lim_{\eta\to0^+}\int f(x)\,\frac{\eta}{x^2+\eta^2}\,dx = f(0)\int_{-\infty}^{\infty}\frac{dt}{1+t^2} = \pi f(0)
\end{equation}
where we have used
\[
    \int _{-\infty} ^{+\infty} \dd t\ 
    \frac{1}{1+t^{2}} 
    =
    \arctan(+\infty) - \arctan(-\infty)
    = \pi.
\]
Thus
\begin{equation}
\frac{\eta}{x^2+\eta^2} \xrightarrow{\ \eta \to 0^+\ } \pi \delta(x).
\end{equation}

\section{Combined result (Sokhotski--Plemelj formula)}

Putting the two limits together:
\begin{equation}
\frac{1}{x-i0^+} = \mathcal{P}\frac{1}{x} + i\pi \delta(x), \qquad
\frac{1}{x+i0^+} = \mathcal{P}\frac{1}{x} - i\pi \delta(x).
\end{equation}

The spelling of the Polish name ``Sochocki'' has some alternatives, 
such as ``Sokhotski''. In the physics literature this is the Dirac relation. 

\printbibliography
\end{document}
